\expandafter\ifx\csname classoptions\endcsname\relax
  \def\classoptions{}
\fi
\documentclass[vyfuk,\classoptions]{fksgeneric}
\usepackage{fkslegacyloader}
\usepackage{framed}
\usepackage{multicol}

\leftheader{Náboj Junior}
\rightheader{IV. ročník \qquad 20. listopadu 2015}

%\setcounter{year}{3}
%\setcounter{batch}{1}

\begin{document}

\section{Užitečné vzorce a~konstanty}

\subsection{Matematika}

\begin{framed}
\begin{multicols}{2}
\vspace{-1.5cm}
\begin{center}
\begin{tabular}{ r l}

Obsah trojúhelníku: & $S\_{trojúhelník} = \frac{1}{2} z v_z$ \\\\
Obvod kruhu: & $o\_{kruh} = 2 \pi r$ \\\\
Obsah kruhu: & $S\_{kruh} = \pi r^{2}$ \\\\
Objem hranolu: & $V\_{hranol} = S\_{podstava} v$ \\\\
Objem kvádru: & $V\_{kvádr} = fgh$\\\\
Pythagorova věta: & $c^{2} = a^{2} + b^{2}$
\end{tabular}
\end{center}

\columnbreak

\begin{center}
\begin{tabular}{ r l }
  
$o$ & obvod \\
$S$ & obsah \\
$V$ & objem \\
$z$ & strana trojúhelníku \\
$v_z$ & výška trojúhelníku na~stranu $z$\\
$r$ & poloměr (kruhu, kružnice) \\
$v$ & výška hranolu \\
$f$, $g$, $h$ & délky hran kvádru\\	
$a$ & přilehlá odvěsna \\
$b$ & protilehlá odvěsna \\
$c$ & přepona\\
$\pi$ & Ludolfovo číslo\\
& $\pi \doteq "3.14" \doteq \dfrac{22}{7}$

\end{tabular}
\end{center}
\end{multicols}

\end{framed}
%%%%%%%%%%%%%%%%%%%%%%%%%%%%%%%%%%%%%%%%%%%%%%%%%%%%%%%%%%%%%%%%%%%%%%%%%%%%%%
\subsection{Mocniny a odmocniny}

\begin{framed}
\begin{multicols}{4}

\begin{center}
\begin{tabular}{l}
$11*11=121$\\
$12*12=144$\\
$13*13=169$\\
$14*14=196$\\
$15*15=225$
\end{tabular}
\end{center}

\columnbreak

\begin{center}
\begin{tabular}{l}
$16*16=256$\\
$17*17=289$\\
$18*18=324$\\
$19*19=361$\\
$20*20=400$
\end{tabular}
\end{center}

\columnbreak

\begin{center}
\begin{tabular}{l}
$21*21=441$\\
$22*22=484$\\
$23*23=529$\\
$24*24=576$\\
$25*25=625$
\end{tabular}
\end{center}

\columnbreak

\begin{center}
\begin{tabular}{l}
$2^2=4$\\
$2^3=8$\\
$2^4=16$\\
$2^5=32$\\
$2^6=64$
\end{tabular}
\end{center}

\end{multicols}\vspace{-\baselineskip}

\begin{multicols}{4}

\begin{center}
\begin{tabular}{l}
$\sqrt{2} \doteq "1.41"$\\
$\sqrt{3} \doteq "1.73"$\\
$\sqrt{5} \doteq "2.24"$
\end{tabular}
\end{center}

\columnbreak

\begin{center}
\begin{tabular}{l}
$\sqrt{6} \doteq "2.45"$\\
$\sqrt{7} \doteq "2.65"$\\
$\sqrt{8} \doteq "2.83"$
\end{tabular}
\end{center}

\columnbreak

\begin{center}
\begin{tabular}{l}
$\sqrt{10} \doteq "3.16"$\\
$\sqrt{11} \doteq "3.32"$\\
$\sqrt{12} \doteq "3.46"$\\
\end{tabular}
\end{center}

\columnbreak

\begin{center}
\begin{tabular}{l}
$\sqrt{13} \doteq "3.61"$\\
$\sqrt{14} \doteq "3.74"$\\
$\sqrt{15} \doteq "3.87"$\\
\end{tabular}
\end{center}

\end{multicols}\vspace{-\baselineskip}
\end{framed}

%%%%%%%%%%%%%%%%%%%%%%%%%%%%%%%%%%%%%%%%%%%%%%%%%%%%%%%%%%%%%%%%%%%%%%%%%%%%%
\subsection{Anglická abeceda}

\begin{framed}
\begin{center}
\begin{tabular}{ccccccccccccc}
A & B & C & D & E & F & G & H & I & J & K & L & M\\
1 & 2 & 3 & 4 & 5 & 6 & 7 & 8 & 9 & 10 & 11 & 12 & 13\\
\midrule
N & O & P & Q & R & S & T & U & V & W & X & Y & Z\\
14 & 15 & 16 & 17 & 18 & 19 & 20 & 21 & 22 & 23 & 24 & 25 & 26
\end{tabular}
\end{center}
\end{framed}

\newpage
%%%%%%%%%%%%%%%%%%%%%%%%%%%%%%%%%%%%%%%%%%%%%%%%%%%%%%%%%%%%%%%%%%%%%%%%%%%%%%
\subsection{Fyzika}

\begin{framed}
\begin{multicols}{2}
\begin{center}
\begin{tabular}{ r l}
 & $W = Fs$ \\ \\
 & $F\_{vz} = \rho\_k V' g$\\ \\
 & $v\_p = \dfrac{s\_{c}}{t\_{c}}$ \\ \\
 & $\rho\_p = \dfrac{m\_c}{V\_c}$ \\ \\
 & Kalorimetrická rovnice:\\
 & $m_1 c_1 (t_1 - t\_k) = Q_1 = Q_2 = m_2 c_2 (t\_k - t_2)$ \\ \\
 & $E\_{k} = \frac{1}{2} m v^{2}$, \qquad $E\_{p} = m g h$\\ \\
 & $p = \frc{F}{S}$ \\ \\
 & $R = \frc{U}{I}$ \\ \\
 & $P = UI$, \qquad $W = Pt$ \\ \\
 & Sériové zapojení odporů:\\
 & $R\_{s} = R_{1} + R_{2} + R_{3} + \dotsi + R_{n}$ \\ \\
 & Paralelní zapojení odporů:\\
 & $\frc{1}{R\_{p}} = \frc{1}{R_{1}} + \frc{1}{R_{2}} + \frc{1}{R_{3}} + \dotsi + \frc{1}{R_{n}}$
\end{tabular}
\end{center}

\columnbreak

\begin{center}
\begin{tabular}{ r l }
$W$ & práce \\
$F$ & síla\\
$s$ & dráha \\
$F\_{vz}$ & vztlaková síla\\
$\rho\_k$ & hustota kapaliny\\
$V'$ & objem ponořené části tělesa\\
$g$ & tíhové zrchlení\\
$v\_p$ & průměrná rychlost \\
$s\_c$ & celková dráha \\
$t\_c$ & celkový čas \\
$\rho\_p$ & průměrná hustota \\
$m\_c$ & celková hmotnost\\
$V\_c$ & celkový objem \\
$Q$ & teplo \\
$c$ & měrná tepelná kapacita \\
$t\_k$ & výsledná teplota \\
$E\_{k}$ & kinetická energie \\
$E\_{p}$ & potenciální energie \\
$m$ & hmotnost\\
$v$ & rychlost\\
$h$ & výška \\
$p$ & tlak \\
$S$ & plocha \\
$R$ & odpor \\
$P$ & elektrický příkon \\
$U$ & napětí \\
$I$ & proud \\
$t$ & čas\\
\end{tabular}
\end{center}

\end{multicols}
\end{framed}

%%%%%%%%%%%%%%%%%%%%%%%%%%%%%%%%%%%%%%%%%%%%%%%%%%%%%%%%%%%%%%%%%%%%%%%%%%%%%%
\subsection{Konstanty}

\begin{framed}
\begin{multicols}{2}

\begin{center}
\begin{tabular}{r l}
tíhové zrychlení & $g = "10 m/s^2"$ \\ \\
hustota vody & $\rho\_{voda} = "1\,000 kg/m^3"$ \\ \\
rychlost zvuku & $v\_z = "300 m/s"$
\end{tabular}
\end{center}

\columnbreak

\begin{center}
\begin{tabular}{r l}
astr. jednotka & $"1 AU" = "150\,000\,000 km"$ \\ \\
rychlost světla & $c\_{světlo} = "300\,000 km/s"$ \\ \\
& $"1 rok" \approx \pi \cdot 10^7" s"$  
\end{tabular}
\end{center}
\end{multicols}
\end{framed}

\end{document}
