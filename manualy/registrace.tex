\expandafter\ifx\csname classoptions\endcsname\relax
  \def\classoptions{}
\fi
\documentclass[vyfuk,\classoptions]{fksempty}

\pagestyle{empty}

\begin{document}
\section{Manuál pro registraci}

Registrace funguje před začátkem samotné soutěže. Je potřeba předem
si připravit stolek a~k~němu vše potřebné. 

\medskip

Připravíte si seznam přihlášených týmů (Documents $\rightarrow$ Team lists),
vytištěný nebo v PC.

\medskip
 
Týmů se ptáte na to, jestli souhlasí údaje v registraci a~jestli se něco nezměnilo.
Změny zanesete buď na papír nebo do počítače. Řeknete jim jejich {\bf číslo},
případně místnost, a aby se šli usadit. Výrazně zdůrazníte,
že \textbf{číslo}, které jste jim řekli, je jejich \textbf{číslo} a~že si mají
hlídat, aby jim vydavači vydávali příklady s tímto \textbf{číslem}.
Poznamenáte si, že jste tento tým už odbavili.

\medskip

Registrace končí buď zaregistrováním všech týmů či uplynutím času.

\medskip

Po registraci {\bf aktualizujte} údaje v databázi (Operators $\rightarrow$ Edit participants),
opět {\bf vygenerujte} a vytiskněte seznam a předejte jej vypisovači diplomů.

\subsection{Manuál pro konec soutěže}
Po konci soutěže obvykle ti samí, co dělali registraci (můžete se dohodnout s někým
jiným, ale někdo to udělat musí), připraví balíček pro ty, co odchází. Balíček se vydává
po vyhodnocení a obsahuje
\begin{compactitem}
	\item diplomy týmům, které nebyly vyhlášené	(diplom mají dostat všichni, ale
prvních několik ho dostane na vyhlášení)
	\item do každého týmu 4 kopie \textbf{potvrzení o účasti}, které se
\textbf{po} Room review vygenerují a vytisknou (Documents $\rightarrow$ Attendance certificates),
	\item do kazdého týmu alespoň jednu brožurku s řešením Náboje Junior.
\end{compactitem}

\end{document}

