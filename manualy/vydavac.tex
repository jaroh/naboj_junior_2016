\expandafter\ifx\csname classoptions\endcsname\relax
  \def\classoptions{}
\fi
\documentclass[vyfuk,\classoptions]{fksempty}

\pagestyle{empty}

\begin{document}
\section{Manuál pro vydavače úloh}

Jakožto vydavač máš vcelku jednoduchou práci. Přijde za tebou 
soutěžící s~papírkem se zadáním úlohy. Pokud na něm bude podpis 
od opravujícího, znamená to, že úloha je správně vyřešena. Papírek si
převezmeš, posuneš ho zadavači a~soutěžícímu dáš novou úlohu.

\medskip

Každý tým má ale svoje úlohy označené 
číslem, a~ty musíš dávat pozor na to, abys dal týmu úlohu, která má 
stejné číslo jako ta, kterou jsi dostal. Důležité tedy je si před začátkem 
soutěže připravit papíry se zadáním na lavici tak, abys mohl rychle 
a~přehledně vydávat nové úlohy. 

V případě, že týmu již nemůžeš vydat novou úlohu (všechny už dostali), poznamenej si, že jsi
měl/a úlohu vydat. V případě, že donesou šestou úlohu, 
za kterou už nic nemůžou dostat, (tj. \textbf{poslední úlohu},
co u~sebe mají), \textbf{co nejdříve ji předej zadavači, aby ji zadal
do systému!}

\end{document}

