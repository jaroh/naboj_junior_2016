\expandafter\ifx\csname classoptions\endcsname\relax
  \def\classoptions{}
\fi
\documentclass[vyfuk,\classoptions]{fksempty}

\pagestyle{empty}

\begin{document}
\section{Manuál pro vedoucího místnosti}

To nejdůležitější, co musí každý vedoucí místnosti zvládat, 
je znalost pravidel, která jsou k~dispozici 
na na wiki a na stránce Náboje.
Dále musí vědět i~to, co konkrétně mají dělat opravovatelé, vydavači atd., a~musí se postarat 
o~to, aby tam všichni potřební organizátoři byli. 
Prostě musí zajistit hladký průběh soutěže ve své místnosti.

\bigskip

Pokud máte více místností, je účelné si vytvořit kromě vedoucích místnosti 
i~vedoucího organizačního místa. 

V případě, že máte aulu či nějakou místnost,
kam se vejdou všechny týmy, pak může některé činnosti (uvítání a vyhlašování)
dělat pouze vedoucí organizačního místa a~ostatní vedoucí místnosti to nemusí řešit, 
jen musí zajistit, aby se tam týmy přesunuly.

\bigskip

Důležitou funkcí vedoucího místnosti je pronést úvodní řeč na začátku soutěže.
Řeč musí obsahovat následující informace:
\begin{compactitem}
	\item Přivítání účastníků.
	\item Zeptat se, jestli každý tým sedí u správného čísla.
	\item Říct, kdo soutěž pořádá - tj. v ČR seminář Výpočty fyzikálních úkolů, součást 
	Matematicko-fyzikální fakulty Univerzity Karlovy
	v~Praze, na SR občanské sdružení Trojsten a konkrétní místo, ve kterém se zrovna nalézá.
	\item Shrnout základní pravidla
		\begin{compactenum}
			\item Soutěž trvá 2 hodiny.
  			\item Každý tým dostane na začátku soutěže 6 úloh, které se snaží vyřešit.
  			\item Pokud si tým myslí, že došel ke správnému řešení,
  					vyšle jednoho zástupce k~opravovateli, který mu řekne,
  					zda-li je řešení špatně, nebo dobře.
  					Zástupce musí předložit {\bf papírek} se zadáním úlohy a s~uvedeným výsledkem.
  			\item Správného opravovatele si zástupce vybere podle dělitelnosti.
  			\item Pokud je řešení špatně,
  					zástupce se vrátí ke svému týmu a počítá dále.
  			\item Pokud je řešení dobře,
  					opravovatel označí papírek (podpisem) se zadáním úlohy 
  					a pošle zástupce k~vydavači, od kterého dostane novou úlohu.
  			\item Cílem týmu je vyřešit co nejvíce příkladů -- v případě, že vyřeší
			všechny, pak co nejdříve.
  			\item Během soutěže jsou promítány aktuální výsledky všech týmů.
  				Ty budou zmrazeny na posledních 20 minut. To znamená, že se do nich
				nepřidávájí nově vyřešené úlohy (neaktualizují se), pouze se
				odpočítává čas do konce soutěže.
			\item Čas řešení se počítá od začátku soutěže po odevzdání poslední úlohy.
		\end{compactenum}
		
	\item Sehrát scénku, která vystihuje základní pravidla (příchod 
	k~opravujícímu, co se děje, pokud je řešení špatně, co, pokud je dobře).
	\item Zodpovědět otázky soutěžících.
	\item Popřát příjemnou soutěž.
\end{compactitem}

\smallskip 

Po úvodní řeči čeká vedoucí místnosti na čas začátku, který se odpočítává
na stránce s výsledkami (Results $\rightarrow$ Live broadcast). Soutěž pak
odstartuje. Globální začátek soutěže znamená, že veškerá organizace musí být
zvládnuta před 10:00.


\medskip 

V~případě dotazů v~průběhu soutěže je vedoucí místnosti aktivně řeší.
Má při sobě i brožurku s řešeními, aby mohl vysvětlovat nejasnosti v úlohách.

V~případě problémů, u~kterých si vedoucí místnosti (či organizačního místa)
není jistý jak je řešit, postupuje v tomto pořadí:
\begin{compactenum}
\item zkontroluje, zda hledaná informace není v pravidlech / brožurce
\item poradí se s delegátem Výfuku
\item zkontaktuje  koordinátora (Patrik Švančara) na tel. čísle {\bf 607 064 672} \\
či mailem (\url{patriksvancara@gmail.com})
\end{compactenum}

Pokud ani celostátní koordinátor neví rovnou, zkonzultuje se Slovenskem a~pak dá vědět.

\smallskip

V průběhu soutěže nezapomeňte pořídit (nechat pořídit) několik fotek místnosti a~řešících studentů.

Pokud se jako vedoucí místnosti nudíte, pomáhejte, kde je potřeba. 
Také kontrolujte, jestli ostatní fungují správně.

\medskip

\subsection{Bezprostředně po soutěži}

\begin{compactitem}
	\item dát krátkou pauzu na WC apod.
	\item rozdat {\bf anketu} a {\bf leták Výfuku}
	\item představit delegáta Výfuku, který odprezentuje Výfuk
	\item přikázat, aby vyplnili anketu
	\item spolu s delegátem Výfuku zabezpečit vysbírání ankety
	\item zorganizovat vyplnění pořadí na diplomy a vytisknout potvrzení
			o účasti (Documents $\rightarrow$ Diplomas)
\end{compactitem}

\subsection{Vyhlašování výsledků}

Při vyhlašování nezapomenout říct a udělat:
\begin{compactitem}
	\item Znovu říct, kdo soutěž pořádá.
	\item Říct, že pokud je baví fyzika a matematika, tak v průběhu roku se
	můžou zapojit do řešení Výfuku či Pikomatu. Zadání Výfuku měli dostat,
	zadání seminářů se obecně dá najít na internetu.
	\item Předat slavnostně odměny odměňovaným v pořadí (minimálně první 3~týmy).
	A~přitom při předávání týmy vyfotit.
	\item Při odchodu dávat i~všem ostatním týmům diplomy a potvrzení o účasti. Tak ať bez něj neutečou
	(jinak je nedostanou).
	\item Poděkovat, že se zúčastnili a~říct, že se na ně a~na jejich mladší spolužáky
	těšíte příští rok :-).
\end{compactitem}

\medskip 

Ke konci vyhlašování dejte pokyn těm, co budou vydávat diplomy, aby se připravili u východu.


\end{document}

