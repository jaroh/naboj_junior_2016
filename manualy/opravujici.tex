\expandafter\ifx\csname classoptions\endcsname\relax
  \def\classoptions{}
\fi
\documentclass[vyfuk,\classoptions]{fksempty}

\pagestyle{empty}

\begin{document}
\section{Manuál pro opravující}

Tvým úkolem jakožto opravujícího bude rozhodovat, jestli je řešení úlohy, 
které ti donese účastník, správně, nebo ne. Abys tuto funkci mohl plnit, 
budeš mít k dispozici papír se správnými odpověďmi a~také by {\bf v~tvé 
blízkosti měla být k~dispozici brožurka s~autorskými řešeními}.

\smallskip

Úlohy jsou označeny číslem. Vás, opravovatelů, je v~místnosti více, 
přičemž každý z~vás má pořadové číslo (od~0 do počtu opravovatelů mínus~1)
a~opravuje v průběhu soutěže pouze úlohy, které po vydělení počtem opravovatelů
dávají zbytek, který odpovídá tvému pořadovému číslu. \textit{Například
uvažujme, že v~místnosti jsou 4~opravovatelé. Jejich čísla jsou 0, 1, 2 a~3.
Opravovatel č.~1 opravuje úlohy 1, 5, 9, 13, atd., opravovatel č.~0 má
úlohy 4, 8, 12, 16, atd.} Aby to účastníci pochopili,
má před sebou každý opravovatel vypsaná čísla úloh, které opravuje (nebo to 
má na tabuli).

Takto to probíhá po většinu soutěže, nicméně se často stává, že musíš zaskakovat 
za jiného opravovatele, protože si šel odskočit nebo se u~jiného 
opravovatele tvoří fronta. V~takových případech můžeš a~měl/a bys opravovat i~jiné úlohy.

\smallskip

Normální průběh je ten, že ti soutěžící přinese papírek se zadáním úlohy, 
na kterém je napsán výsledek. Ty výsledek zkontroluješ a~pokud je špatně, uděláš čárku 
vedle čísla úlohy a~tým pošleš zpátky počítat. Pokud je výsledek dobře, označíš papírek 
se zadáním (razítkem, podpisem) a~pošleš soutěžícího za vydavačem. V~případě, že má
na papírku u~čísla úlohy již 3 a~více čárek, pak se můžeš před označením papírku 
týmu zeptat na postup řešení 
(má to sloužit k tomu, aby pořád netipovali).

\smallskip

Buď tolerantní. Pokud se výsledek liší o~desetinu nebo je zaokrouhlený, uznej ho. 
Také vždy zkontroluj, že se výsledek, který ti soutěžící donese, neliší od tvého 
{\bf pouze algebraickou úpravou}. Většinou ale požadujeme číselné výsledky.

Pozor na problematické věci při opravování (tj. dívej se pořádně, jestli to mají
dobře -- někdy to mají dobře, když ti přijde, že ne, a někdy to mají špatně, i když to na 
první pohled vypadá, že je to dobře).
\begin{compactitem}
	\item Převody jednotek (litry na metry krychlové, sekundy na hodiny...)
	\item Řády výsledků ($100000000$ není $10\,000\,000$, ale je to $100\,000\,000$)
\end{compactitem}

\smallskip

Občas se stane, že máme výsledek špatně (byť doufejme, že ne). Proto, pokud 
k~tobě bude chodit hodně týmů se stejným výsledkem a~budou se tvářit hodně přesvědčivě, 
že to mají dobře, dej vědět vedoucímu místnosti, který to začne řešit na mezinárodní úrovni.

\end{document}
