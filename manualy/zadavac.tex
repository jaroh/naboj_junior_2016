\expandafter\ifx\csname classoptions\endcsname\relax
  \def\classoptions{}
\fi
\documentclass[vyfuk,\classoptions]{fksempty}

\pagestyle{empty}

\begin{document}
\section{Manuál pro zadavače vyřešených úloh}

\noindent
Jako zadavač se staráš o technické věci (projektor a administrátorské
rozhraní). Máš na starosti:

\begin{compactitem}
	\item ovládání projektoru a~že se na něm promítají v~průběhu soutěže aktuální výsledky,
	\item kontrolu e-mailové konference \url{junior-mesta@vyfuk.mff.cuni.cz} (měl bys v ní být),
	\item v~průběhu soutěže zadáváš do systému kódy úloh,
	\item po soutěži uděláš \textbf{Room review},
	\item pak pomůžeš při tvorbě diplomů (v admin. rozhraní máš nezakrytou výsledkovou listinu).
\end{compactitem}

\medskip

\subsection{IT support aneb co udělat hned ráno}
\begin{compactenum}
	\item Přihlaš se na \url{junior.naboj.org/admin}. Pokud by ti nefungovalo
přihlášení, ihned kontaktuj centrálu (607 064 672, Patrik Švančara).
	\item Otevři si mail. Cca 30 min před začátkem ti přijde mail, na který pak odpovíš, abychom věděli, kdo je online.
	\item Vyzkoušej, zda-li máš v prohlížeči povolen Javascript, a zda-li ti funguje stránka Operators $\rightarrow$ Barcode scanning.
	\item Vygeneruj aktuální seznam týmů (Documents $\rightarrow$ Team lists),
tedy klikni u svého města na Generate a pak na link vedoucí k pdf. Seznam
vytiskni / dones na notebooku / jinak dodej registrátorům.
	\item Možná se změní nějaká jména, resp. ročníky u soutěžících.
Všechny změny aktualizuj pomocí Operators $\rightarrow$ Edit participants.
	\item Znovu {\bf vygeneruj} Team list a předej ho člověku, který bude
vypisovat diplomy.
\end{compactenum}

\subsection{Live výsledky}
Před soutěží zapoj projektor a spáruj jej s PC. Do okna projektoru umísti prohlížeč
s výsledkami (Results $\rightarrow$ Live broadcast) na fullscreen. Měl by tam běžet
odpočet času do začátku Náboje. \\


\noindent
Posledních 15 minut je výsledkovka zmrazená, tzn. neaktualizuje se, ale pořád
se odpočítává čas do konce soutěže, takže projektor nevypínej. \\

\noindent
Ty můžeš výsledky vidět v části Results $\rightarrow$ Admin secret.

\subsection{Zadávání úloh}
Úlohy ti bude podávat vydavač. Doporučeno je zadávat do systému úlohy
{\bf ve stejném pořadí}, jak je týmy nosí. Po zadání do systému si úlohy u sebe
nechej a průběžně je rozděluj podle týmů (není to nutné, ale v případě problémů
na konci soutěže si usnadníte práci).

\bigskip

\noindent
Zadávání výsledků se děje pomocí aplikace Barcode scanning v sekci Operators.

\smallskip

\noindent
Nejdříve vybereš prefix své místnosti. Je to to číslo, na které začínají
všechny kódy týmů v místnosti (případně jsou to první (malé) číři číslice
na všech úlohách, které máte v místnosti). Toto číslo již do systému nenahazuješ.

\smallskip

\noindent
Samotné zadávání funguje tak, že do pole Barcode \emph{opíšeš} pětimístné
číslo napsané velkými čísly na boku úlohy. Pak stiskneš enter nebo klikneš
na submit. Dole se začne vytvářet tabulka se zadanými kódy. Vždy zkontroluj,
jestli jsi zadal správný kód (poslední číslice kódu je číslice pro kontrolní
součet, \uv{náhodné} zadání jiného platného kódu je tak minimální).
Pokud jsi udělal chybu, klikni na šipku zpět.

\subsection{Room review}
Po skončení soutěže a zadání všech vyřešených úloh musíš udělat Room review
(Operators $\rightarrow$ Room review):

\begin{compactenum}
	\item Vyber svou místnost kliknutím na lupičku.
	\item Postupně zadávej kódy {\bf prvních úloh, které jste týmům nevydali}. Pokud jste vydali nějakému týmu všechny
úlohy, zadej jakoby kód úlohy s pořadovým číslem 0.
	\item V případě problémů s týmem se podívej na instrukce na stránce, resp. klikni na lupičku u problémového týmu.
Zobrazí se ti podrobné info o nahraných úlohách.
	\item Po zezelenání všech týmů jsou výsledky kompletní.
	\item Výsledky \textbf{až po této kontrole} předej vypisovači diplomů, aby do nich napsal umístění.
	\item Vygeneruj rovněž potvrzení o účasti (Documents $\rightarrow$ Diplomas)
a zabezpeč jeho tisk. Potvrzení by měl získat každý účastník,
	proto vygenerovaný soubor nezapomeňte vytisknout 4-krát.
\end{compactenum}

\subsection{Zobrazování výsledků}
Pro napínavé vyhlašování výsledků slouží Results $\rightarrow$ Final announcements. Zde
si nastavíš, kolik týmů má být skrytých a pohybem šipky nahoru je postupně
odhaluješ. Toto můžete využít při vyhlašování vítězů, není to ale nutnost.
(Před samotným vyhlašováním doporučujeme zkontrolovat funkčnost této stránky.)

\subsection{Po vyhlášení výsledků}
Přihlaš se do admin. prostředí a na úvodní stránce zaznač v sloupci Announced,
že výsledky byly vyhlášeny. Tím tvoje práce končí :).
\end{document}
