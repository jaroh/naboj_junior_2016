{%format uvodnicku
\setlength{\parskip}{10pt}%
\setlength{\parindent}{0pt}%
\setlength{\baselineskip}{1.2\baselineskip}%
%%%%%%%%%%%%%%%%%%%%%%%%%%%%%%%%%%%%%%%%%%%%%%%%%%%%%%%%%%%%%%%%%%%%%%%%%
%%%%%%%%%%%%%%%%%%%%%%%%%  CESKA VERZE  %%%%%%%%%%%%%%%%%%%%%%%%%%%%%%%%%
%%%%%%%%%%%%%%%%%%%%%%%%%%%%%%%%%%%%%%%%%%%%%%%%%%%%%%%%%%%%%%%%%%%%%%%%%
\opt{lngcs}{%
\noindent
{\normalfont\sffamily\slshape\Large Milí příznivci matematiky a~fyziky,}

\medskip\noindent
v~rukou držíte brožurku čtvrtého ročníku soutěže Náboj Junior,
ve které naleznete zadání a~vzorová řešení 42~úloh této soutěže.
Náboj Junior je týmová soutěž v~řešení matematických a~fyzikálních
problémů určená pro žáky druhého stupně základních škol a~odpovídajících
ročníků víceletých gymnázií. Spoluvyhlašovateli soutěže jsou Ministerstvo
školství, mládeže a~tělovýchovy České republiky a~Matematicko-fyzikální
fakulta Univerzity Karlovy v~Praze.

Letos Náboj Junior probíhal současně na~třinácti místech České republiky
a~na~šestnácti místech Slovenska. Veškeré informace o~průběhu soutěže,
včetně mezinárodních výsledků, jsou k~nalezení na
stránce~\url{junior.naboj.org}. Pokud vás tato soutěž
zaujala, jistě budete potěšeni zprávou, že v~příštím roce proběhne další
ročník.

Pokud byste chtěli uspořádat regionální kolo i~ve vašem městě a~zvýšit tak
přístupnost soutěže v~regionu, budeme velice potěšeni a~rádi s~vámi navážeme
spolupráci. V~případě zájmu nám napište na kontaktní e-mail.

Na organizaci soutěže se podíleli organizátoři a~přátelé korespondenčního
semináře Výfuk, který zastřešuje Matematicko-fyzikální fakulta Univerzity
Karlovy v~Praze, ve spolupráci s~jednotlivými organizačními místy.
Na Slovensku organizaci zabezpečilo občanské sdružení Trojsten.
\par\noindent
Přejeme vám příjemné rozjímání nad příklady,

\hfill\sign{Organizátoři}{info-cz@junior.naboj.org}

% jazykove korektury OK -- Radka, Tom

}%%%%%%%%%%%%%%%%%%%%%%%%%%%%%%%%%%%%%%%%%%%%%%%%%%%%%%%%%%%%%%%%%%%%%%%%
%%%%%%%%%%%%%%%%%%%%%%%  SLOVENSKA VERZE  %%%%%%%%%%%%%%%%%%%%%%%%%%%%%%%
%%%%%%%%%%%%%%%%%%%%%%%%%%%%%%%%%%%%%%%%%%%%%%%%%%%%%%%%%%%%%%%%%%%%%%%%%
\opt{lngsk}{%
\noindent
{\normalfont\sffamily\slshape\Large Ahojte,}

\noindent\medskip
práve sa Vám do rúk dostala brožúrka zadaní a~riešení úloh Náboja Junior~2015. 

Náboj Junior je matematicko-fyzikálna súťaž pre štvorčlenné tímy žiakov druhého stupňa základnej
školy a~žiakov nižšieho stupňa viacročných gymnázií. Celá súťaž trvá presne~120~minút,
počas ktorých sa tímy snažia vyriešiť čo najviac úloh.

V~tejto súťaži nejde o~bezhlavú aplikáciu už nadobudnutých vedomostí. Úlohy
vyžadujú tiež istú dávku invencie a~dôvtipu.

4.~ročník Náboja Junior prebiehal dňa~20.\,11.\,2015 v~šestnástich mestách
na Slovensku a~v~trinástich mestách v~Českej republike.
Práve v~týchto mestách sa našli šikovní organizátori zo stredných, prípadne
vysokých škôl a~umožnili základným školám z~regióna si zasúťažiť
a~preveriť svoje vedomosti.

Cieľom súťaže je ukázať, že matematika a~fyzika sú zaujímavé prírodné vedy
s~množstvom výziev a~príležitostí pre každého žiaka. Zároveň organizátori
dostanú možnosť vytvoriť svoju súťaž a~zistiť, že organizovanie a~práca
v~tíme vie byť zábavná, ale aj náročná.

Súťaž Náboj Junior vznikla ako spoločná aktivita občianskeho združenia Trojsten
a~korešpondenčného seminára~MFF~UK~Výfuk. Členovia organizácií sú
vysokoškolskí študenti Fakulty matematiky, fyziky a~informatiky~UK
v~Bratislave alebo Ma\-te\-ma\-ticko-fyzikální fakulty~UK v~Prahe,
ktorí sa snažia o~rozvoj detí a~študentov. 
\par\noindent
Prajeme príjemné zamýšľanie sa nad príkladmi,

\hfill\sigstyle{o.~z.~Trojsten a~korešp. seminár MFF UK Výfuk}
}%lngsk
}%format
