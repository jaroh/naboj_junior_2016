% Neodmazavejte ty % jinak vam Donald ruce ukousne!!!
%
% Bylo by super, když se úloha zásadně přejmenuje, tak přejmenovat
% i její soubor a opravit to v problem-sort.tex
%
\uloha{\opt{lngcs}{ciferný součin}\opt{lngsk}{}}% název s malým písmenem (velké u vl. jmen)
{\opt{lngcs}{% zadání CZ
Lukáše by zajímalo, kolik existuje šesticiferných čísel, jejichž ciferný součin
(součin všech cifer, ze kterých číslo skládá) je roven $750$. Kolik?
}%
\opt{lngsk}{% zadanie SK
}}%
{Celkem existuje $180$ takových čísel.}% výsledek ve všech možných formách včetně veličiny
{}% autor vzoráku~--~typicky prijmeni (jinak o tom vis), vsechna mala
{\opt{lngcs}{% řešení CZ
Prvočíselný rozklad čísla $720$ je $2 \cdot 3 \cdot 5 \cdot 5 \cdot 5$.
Naše šesticiferné číslo má tedy cifry z množiny $\{1;2;3;5;5;5\}$ anebo
$\{1;1;6;5;5;5\}$ a my akorát musíme určit počet všech možností, jak tyto
cifry uložit na jednotlivé pozice.

První tři cifry z první množiny lze uložit na pozice ve výsledném číslu
$6 \cdot 5 \cdot 4 = 120$ způsoby. Zbylé pozice pak musíme obsadit pětkami, tzn.
pouze jedním způsobem. Stejně můžeme postupovat i v druhé množině.
Nejdříve si v ní ale \uv{obarvíme} jedničky různou barvou a dostaneme tak
rovněž $120$ různých čísel. V nich ale vždy nalezneme dvojice, jež mají jedničky
na stejných pozicích, ale s přehozenými barvami. Po myšlenkovém \uv{odbarvení}
jedniček nám tak zůstane pouze $60$ různých čísel. Ve finále tak dostaneme, že
možných šesticiferných čísel je $120 + 60 = 180$.
}%
\opt{lngsk}{% riešenie SK
}}%
{\opt{lngcs}{}\opt{lngsk}{}}% pôvod úlohy
