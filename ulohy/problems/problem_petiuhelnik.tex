% Neodmazavejte ty % jinak vam Donald ruce ukousne!!!
%
% Bylo by super, když se úloha zásadně přejmenuje, tak přejmenovat
% i její soubor a opravit to v problem-sort.tex
%
\uloha{\opt{lngcs}{nepravidelný pětiúhelník}\opt{lngsk}{}}% název s malým písmenem (velké u vl. jmen)
{\opt{lngcs}{% zadání CZ
Terka dostala od Radky hádanku: jaký je obsah nakresleného pětiúhelníku?
Zadané jsou jeho čtyři strany $a$, $b$, $c$ a $e$, přičemž $a = c$, a tři pravé úhly
(viz obrázek). Výsledek uveďte v nejkratším možném tvaru.
}%
\opt{lngsk}{% zadanie SK
}}%
{Obsah pětiúhelníku je $ae$ (uznejte i jakýkoliv jiný ekvivalentní algebraický tvar).}% výsledek ve všech možných formách včetně veličiny
{uhlirova}% autor vzoráku~--~typicky prijmeni (jinak o tom vis), vsechna mala
{\opt{lngcs}{% řešení CZ
Útvar si rozdělíme na obdélník a dva trojúhelníky, jejichž obsahy postupně 
spočteme (viz obrázek). Obsah obdélníku je $S_1 = ab$. Obsah pravoúhlého trojúhelníku
nad obdélníkem je $S_2 = a(e-b)/2$.

Pro výpočet obsahu posledního trojúhelníku potřebujeme zjistit délku strany $d$, 
k čemuž dospějeme pomocí dvou Pythagorových vět (přeponu značíme $x$):
\eq[m]{
  a^2 + \(e - b\)^2 &= x^2\,,\\
  d^2 + c^2 & = x^2\,,
}
Porovnáním levých stran rovnic dostaneme
\eq{
  d = \sqrt{a^2 + \(e - b\)^2 - c^2} = \sqrt{a^2 + \(e - b\)^2 - a^2} = e - b\,.
}
Obsah trojúhelníku tedy vychází $S_3 = a(e-b)/2$. Celkový obsah získáme součtem
obsahů dílčích útvarů:
\eq{
  S = ab + \frac{a\(e - b\)}{2} + \frac{a\(e - b\)}{2} = ab + a\(e - b\) = ab + ae - ab = ae\,.
}
Obsah Terčina pětiúhelníku je $ae$.
}%
\opt{lngsk}{% riešenie SK
}}%
{\opt{lngcs}{}\opt{lngsk}{}}% pôvod úlohy
