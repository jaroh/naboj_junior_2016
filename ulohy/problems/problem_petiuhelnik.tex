% Neodmazavejte ty % jinak vam Donald ruce ukousne!!!
%
% Bylo by super, když se úloha zásadně přejmenuje, tak přejmenovat
% i její soubor a opravit to v problem-sort.tex
%
\uloha{\opt{lngcs}{nepravidelný pětiúhelník}\opt{lngsk}{}}% název s malým písmenem (velké u vl. jmen)
{\opt{lngcs}{% zadání CZ
Terka dostala od Radky hádanku: jaký je obsah nakresleného pětiúhelníku?
Zadané jsou jeho čtyři strany $a$, $b$, $c$ a $e$, přičemž $a = c$, a tři pravé úhly
(viz obrázek). Výsledek uveďte v nejkratším možném tvaru.
}%
\opt{lngsk}{% zadanie SK
}}%
{Obsah pětiúhelníku je $"ae"$ (uznejte i jakýkoliv jiný ekvivalentní algebraický tvar).}% výsledek ve všech možných formách včetně veličiny
{uhlirova}% autor vzoráku~--~typicky prijmeni (jinak o tom vis), vsechna mala
{\opt{lngcs}{% řešení CZ
Útvar si rozdělíme na obdélník a dva trojúhelníky, jejichž obsahu postupně 
spočteme (viz obrázek).
Obsah obdélníku je $S_1 = ab$. Obsah trojúhelníku nad obdélníkem je $S_2 = a(e-b)/2$.
Pro výpočet obsahu posledního trojúhelníku potřebujeme zjistit délku strany $d$, 
k čemuž dospějeme pomocí dvou pythagorových vět (přeponu značíme $x$):
$a^2 + (e-b)^2 = x^2$ a $x^2 = d^2 + c^2$, tedy $d = sqrt(a^2 + (e-b)^2 - c^2)
= sqrt(a^2 + (e-b)^2 - a^2) = (e-b)$. Obsah trojúhelníku tedy vychází $S_3 = a(e-b)/2$.
Celkový obsah získáme jakou součet dílčích: $S = ab + a(e-b)/2 + a(e-b)/2 = ae$.
}%
\opt{lngsk}{% riešenie SK
}}%
{\opt{lngcs}{}\opt{lngsk}{}}% pôvod úlohy
