% Neodmazavejte ty % jinak vam Donald ruce ukousne!!!
%
% Bylo by super, když se úloha zásadně přejmenuje, tak přejmenovat
% i její soubor a opravit to v problem-sort.tex
%
\uloha{\opt{lngcs}{megaskluzavka}\opt{lngsk}{}}% název s malým písmenem (velké u vl. jmen)
{\opt{lngcs}{% zadání CZ
Fyzici mají někdy zvláštní nápady. Onehdy vymysleli megaskluzavku, ze které
pouští krychli s rozměry $"2 m" \times "2 m" \times "2 m"$ z výšky $"10 m"$
vzhledem k jejímu středu (viz obrázek) po čtvrtkružnici na vodorovnou
dráhu (viz obrázek). Po namydlené skluzavce se krychle pohybuje bez tření,
ale na vodorovné dráze je činitel tření $"0.2"$. Jakou 
vzdálenost po vodorovné dráze krychle ujede, dokud se nezastaví?
}%
\opt{lngsk}{% zadanie SK
}}%
{Krychle se zastaví po $"45 m"$.}% výsledek ve všech možných formách včetně veličiny
{hofierka}% autor vzoráku~--~typicky prijmeni (jinak o tom vis), vsechna mala
{\opt{lngcs}{% řešení CZ
Úloha se řeší pomocí zákona zachování energie. Potenciální energie krychle se nejprve přemění na kinetickou energii a pak se tato energie uvolní ve formě tepla v procesu tření.

Postačuje nám uvažovat rovnost celkové energie na začátku a na druhé strane práci vykonanou třecími sílami. Máme tedy:
 \eq{
  E_p = m g h = W_t = F_t \cdot s = m g f s,
}
kde $f$ je činitel tření, $h$ je změna výšky těžistě, $s$ je vzdálenost, po které působí tření a $m$ je hmotnost krychle. Využili jsme, že pro trecí sílu je $F_t=F_n f$, kde normálová síla $F_n$ splňuje $F_n=F_g$.

Odtud vyjádřime veličinu $s$ a máme $"s=\frac{h}{f}=\frac{9 m}{0,2}=45 m"$. Krychle se zastaví po $"45 m"$ na vodorovné drázes.

}%
\opt{lngsk}{% riešenie SK
}}%
{\opt{lngcs}{}\opt{lngsk}{}}% pôvod úlohy
