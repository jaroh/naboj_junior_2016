% Neodmazavejte ty % jinak vam Donald ruce ukousne!!!
%
% Bylo by super, když se úloha zásadně přejmenuje, tak přejmenovat
% i její soubor a opravit to v problem-sort.tex
%
\uloha{\opt{lngcs}{megaskluzavka}\opt{lngsk}{}}% název s malým písmenem (velké u vl. jmen)
{\opt{lngcs}{% zadání CZ
Fyzici mají někdy zvláštní nápady. Onehdy vymysleli megaskluzavku, ze které
pouští krychli s rozměry $"2 m" \times "2 m" \times "2 m"$ z výšky $"10 m"$
vzhledem k jejímu středu (viz obrázek) po čtvrtkružnici na vodorovnou
dráhu (viz obrázek). Po namydlené skluzavce se krychle pohybuje bez tření,
ale na vodorovné dráze je činitel tření $"0.2"$. Jakou 
vzdálenost po vodorovné dráze krychle ujede, dokud se nezastaví?
}%
\opt{lngsk}{% zadanie SK
}}%
{Krychle se zastaví po $"45 m"$.}% výsledek ve všech možných formách včetně veličiny
{}% autor vzoráku~--~typicky prijmeni (jinak o tom vis), vsechna mala
{\opt{lngcs}{% řešení CZ
}%
\opt{lngsk}{% riešenie SK
}}%
{\opt{lngcs}{}\opt{lngsk}{}}% pôvod úlohy
