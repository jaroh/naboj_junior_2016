% Neodmazavejte ty % jinak vam Donald ruce ukousne!!!
%
% Bylo by super, když se úloha zásadně přejmenuje, tak přejmenovat
% i její soubor a opravit to v problem-sort.tex
%
\uloha{\opt{lngcs}{Krychle}\opt{lngsk}{}}% název s malým písmenem (velké u vl. jmen)
{\opt{lngcs}{% zadání CZ
Mějme krychli $\bod{ABCDEFGH}$, přičemž bod $\bod{X}$ je střed stěny $\bod{ABCD}$,
bod $\bod{K}$ je střed strany $\bod{BC}$, bod $\bod{L}$ je střed strany $\bod{CD}$
a bod $\bod{M}$ je střed strany $\bod{FG}$.
Dále označíme průsečík přímky $\bod{EX}$ a roviny $\bod{KLM}$ jako bod $\bod{P}$.
Jaká je vzdálenost bodu $\bod{P}$ od těžiště krychle, jestliže délka její strany je $a$?
%TODO obrázek krychle
}%
\opt{lngsk}{% zadanie SK
}}%
{}% výsledek ve všech možných formách včetně veličiny
{}% autor vzoráku~--~typicky prijmeni (jinak o tom vis), vsechna mala
{\opt{lngcs}{% řešení CZ
}%
\opt{lngsk}{% riešenie SK
}}%
{\opt{lngcs}{}\opt{lngsk}{}}% pôvod úlohy
