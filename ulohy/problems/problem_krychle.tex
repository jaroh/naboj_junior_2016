% Neodmazavejte ty % jinak vam Donald ruce ukousne!!!
%
% Bylo by super, když se úloha zásadně přejmenuje, tak přejmenovat
% i její soubor a opravit to v problem-sort.tex
%
\uloha{\opt{lngcs}{krychle}\opt{lngsk}{}}% název s malým písmenem (velké u vl. jmen)
{\opt{lngcs}{% zadání CZ
Mějme krychli $\bod{ABCDEFGH}$, přičemž bod $\bod{X}$ je střed stěny $\bod{ABCD}$,
bod $\bod{K}$ je střed strany $\bod{BC}$, bod $\bod{L}$ je střed strany $\bod{CD}$
a bod $\bod{M}$ je střed strany $\bod{FG}$.
Dále označíme průsečík přímky $\bod{EX}$ a roviny $\bod{KLM}$ jako bod $\bod{P}$.
Jaká je vzdálenost bodu $\bod{P}$ od těžiště krychle, jestliže délka její strany je $a$?
}%
\opt{lngsk}{% zadanie SK
}}%
{Vzdálenost bodu $\bod{P}$ od těžiště krychle je $a\sqrt{9/8} = 3a\sqrt{2}/4$.}% výsledek ve všech možných formách včetně veličiny
{}% autor vzoráku~--~typicky prijmeni (jinak o tom vis), vsechna mala
{\opt{lngcs}{% řešení CZ
Podíváme-li se na situaci seshora (viz obrázek), situace se zjednoduší, pač
seshora vidíme pouze posunutí ve vodorovném směru. Ihned si můžeme všimnout,
že vodorovná vzdálenost průsečníku $\bod{EX}$ a roviny $\bod{KLM}$, tzn. bodu
$\bod{P}$, od středu krychle je rovna jedné čtvrtine uhlopříčky. Délka
uhlopříčky je z Pythagorovy věty $u = \sqrt{a^2 + a^2} = a \sqrt{2}$, takže
její čtrtina je $x = u/4 = a\sqrt{2}/4$.

Podíváme-li se na situaci zboku, můžeme si všimnout, že trojúhelníky vymezené
stranami $\bod{AE}$, $\bod{AB}$, přímkou $\bod{EX}$ a bodem $\bod{P}$ mají
všechny úhly shodné, takže jsou si podobné a poměr jejich stran je rovný ${1:2}$.
Odtud určíme, že svislá vzdálenost bodu $\bod{P}$ od úsečky $\bod{AB}$ je
$a/2$, a tedy svislá vzdálenost tohoto bodu od středu krychle (jejího těžiště)
je $y = a/2 + a/2 = a$.

Přímou vzdálenost bodu $\bod{P}$ od středu krychle určíme z Pythagorovy věty,
protože již známe jeho vodorovnou ($x$ a svislou vzdálenost ($y$):
\eq{
  d = \sqrt{x^2 + y^2} = \sqrt{\(\frac{a\sqrt{2}}{4}\)^2 + a^2} = \sqrt{\frac{2a^2}{16} + \frac{16a^2}{16}} = \sqrt{\frac{18a^2}{16}} = \frac{3a\sqrt{2}}{4}\,.
}
Bod $\bod{P}$ je vzdálen od těžiště krychle o vzdálenost $3a\sqrt{2}/4$.
}%
\opt{lngsk}{% riešenie SK
}}%
{\opt{lngcs}{}\opt{lngsk}{}}% pôvod úlohy
% TODO obrázek krychle do zadání
% TODO obrázek krychle seshora a zboku do vzoráku
