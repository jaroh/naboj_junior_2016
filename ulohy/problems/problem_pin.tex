% Neodmazavejte ty % jinak vam Donald ruce ukousne!!!
%
% Bylo by super, když se úloha zásadně přejmenuje, tak přejmenovat
% i její soubor a opravit to v problem-sort.tex
%
\uloha{\opt{lngcs}{zapomenutý PIN}\opt{lngsk}{}}% název s malým písmenem (velké u vl. jmen)
{\opt{lngcs}{% zadání CZ
David opět něco zapomněl, tentokrát svůj PIN ke kreditní kartě. Pamatuje si však,
že PIN je čtyřmístný, na druhém místě je číslice $8$ a na posledním číslice $9$. Také
ví, že celé číslo je dělitelné devíti a číslice se neopakují. Kolik různých 
možností existuje pro jeho možný PIN?
}%
\opt{lngsk}{% zadanie SK
}}%
{Pro Davidův PIN existuje $6$ různých možností.}% výsledek ve všech možných formách včetně veličiny
{}% autor vzoráku~--~typicky prijmeni (jinak o tom vis), vsechna mala
{\opt{lngcs}{% řešení CZ
}%
\opt{lngsk}{% riešenie SK
}}%
{\opt{lngcs}{}\opt{lngsk}{}}% pôvod úlohy
