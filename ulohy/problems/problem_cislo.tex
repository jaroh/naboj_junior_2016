% Neodmazavejte ty % jinak vam Donald ruce ukousne!!!
%
% Bylo by super, když se úloha zásadně přejmenuje, tak přejmenovat
% i její soubor a opravit to v problem-sort.tex
%
\uloha{\opt{číselný trojúhelník}{}\opt{lngsk}{}}% název s malým písmenem (velké u vl. jmen)
{\opt{lngcs}{% zadání CZ
Do trojúhelníku na obrázku jsou doplňována přirozená čísla tak, aby součin třech
čísel na každé jeho straně byl stejný. Jaké nejmenší číslo může patřit do
šedého kolečka?}%
\opt{lngsk}{% zadanie SK
}}%
{Nejmenší číslo je $21$.}% výsledek ve všech možných formách včetně veličiny
{uhlirova}% autor vzoráku~--~typicky prijmeni (jinak o tom vis), vsechna mala
{\opt{lngcs}{% řešení CZ
Čísla v kolečkách si rozložíme na prvočinitele. Tím zjistíme, že na levou stranu
musíme doplnit dvě trojky a napravou jednu dvojku, aby součiny čísel na každé z těchto
dvou stran si byly rovny. Jedná se o součin čísel 2,2,2,2,2,3,3,3,7. Do prostředního
kolečka nám tedy zbývá umístit čísla 2,2,2,2,3 a 7. Jelikož však hledáme nejmenší
možné číslo, všimneme si, že můžeme po dvou dvojkách přidat do každého z krajních
koleček. Tím se však součin čísel na stranách zčtyřnásobí. Proto do těchto krajních
koleček přidáme ještě po čísle 4, čímž se součiny levé a pravé straně
celkem zšestnáctinásobí. To však nevadí, neboť i součin na spodní straně se přidáním
dvou čtyřek zšestnáctinásobil. V protředním kolečku nám tedy zbylo číslo 21.
}%
\opt{lngsk}{% riešenie SK
}}%
{\opt{lngcs}{}\opt{lngsk}{}}% pôvod úlohy
