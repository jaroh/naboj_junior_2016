% Neodmazavejte ty % jinak vam Donald ruce ukousne!!!
%
% Bylo by super, když se úloha zásadně přejmenuje, tak přejmenovat
% i její soubor a opravit to v problem-sort.tex
%
\uloha{\opt{číselný trojúhelník}{}\opt{lngsk}{}}% název s malým písmenem (velké u vl. jmen)
{\opt{lngcs}{% zadání CZ
Do trojúhelníku na obrázku jsou doplňována přirozená čísla tak, aby součin třech
čísel na každé jeho straně byl stejný. Jaké nejmenší číslo může patřit do
šedého kolečka?}%
\opt{lngsk}{% zadanie SK
}}%
{Nejmenší číslo je $21$.}% výsledek ve všech možných formách včetně veličiny
{uhlirova}% autor vzoráku~--~typicky prijmeni (jinak o tom vis), vsechna mala
{\opt{lngcs}{% řešení CZ
Čísla v kolečkách si rozložíme na prvočinitele. Zjistíme tím, že na levé a pravé
straně trojúhelníka spolu násobíme jiný počet různých prvočísel. Aby byly součiny
stran stejné, musíme na levou stranu doplnit alespoň dvě trojky a na pravou stranu
alespoň jednu dvojku. Tím dostaneme na každé straně součin
$2 \cdot 2 \cdot 2 \cdot 2 \cdot 2 \cdot 3 \cdot 3 \cdot 3 \cdot 7$.

Do prostředního kolečka nám tedy zbývá umístit číslo, které získáme součinem
zbylých čísel, tzn. $2 \cdot 2 \cdot 2 \cdot 2 \cdot 3 \cdot 7$. Jelikož však
hledáme nejmenší možné číslo, všimneme si, že můžeme po dvou dvojkách přidat
do každého z krajních koleček a rovnost součinů se nezmění. V prostředním
kolečku nám tedy zbyde číslo $3 \cdot 7 = 21$.
}%
\opt{lngsk}{% riešenie SK
}}%
{\opt{lngcs}{}\opt{lngsk}{}}% pôvod úlohy
