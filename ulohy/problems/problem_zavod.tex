% Neodmazavejte ty % jinak vam Donald ruce ukousne!!!
%
% Bylo by super, když se úloha zásadně přejmenuje, tak přejmenovat
% i její soubor a opravit to v problem-sort.tex
%
\uloha{\opt{lngcs}{australský závod}\opt{lngsk}{}}% název s malým písmenem (velké u vl. jmen)
{\opt{lngcs}{% zadání CZ
Pes dingo chce ulovit malého klokana. Připlíží se k němu na vzdálenost
$"26 m"$ a vyběhne. Zatímco pes uběhne $"5 m"$, stihne klokan skočil dvakrát,
Průměrná délka skoku je $"2 m"$. Kolik metrů musí dingo běžet, aby klokana
ulovil?
}%
\opt{lngsk}{% zadanie SK
}}%
{Dingo musí uběhnout $"130 m"$.}% výsledek ve všech možných formách včetně veličiny
{uhlirova}% autor vzoráku~--~typicky prijmeni (jinak o tom vis), vsechna mala
{\opt{lngcs}{% řešení CZ
Počáteční vzdálenost psa dinga a klokana označíme $"s_0 = 26 m"$, uběhlou dráhu
psa dinga $s_D$ a rychlost psa dinga $v_D$ a klokana $v_K$. Za čas $t$ doběhne pes
dingo klokana a uběhne tak vzdálenost $s_D = v_D t$. Zároveň však pro tuto vzdálenost
platí $s_D = s_0 + v_K t$. Za čas $t$ v druhé rovnici dosadíme z rovnice první, čímž
získáme $s_D = s_0 + v_K s_D/ v_D$, tedy $s_D \left(1 - v_K / v_D) = s_0$.
Poměr rychlostí klokana a psa dinga získáme snadno: $v_K / v_D = 5 / 4$.
Pes dingo tedy musí uběhnout vzdálenost $"s_D = 26 \cdot 5 = 130 m"$.
}%
\opt{lngsk}{% riešenie SK
}}%
{\opt{lngcs}{}\opt{lngsk}{}}% pôvod úlohy
