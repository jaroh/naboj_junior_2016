% Neodmazavejte ty % jinak vam Donald ruce ukousne!!!
%
% Bylo by super, když se úloha zásadně přejmenuje, tak přejmenovat
% i její soubor a opravit to v problem-sort.tex
%
\uloha{\opt{lngcs}{cesta do práce}\opt{lngsk}{}}% název s malým písmenem (velké u vl. jmen)
{\opt{lngcs}{% 
Patrik jezdí každé ráno do práce autem. V následujících grafech můžete vidět
závislost jeho rychlosti $v$ na čase a dráhy $s$ na čase $t$. Určete průměrnou
rychlost Patrikovy cesty.
}%
\opt{lngsk}{% zadanie SK
}}%
{Průměrná rychlost je $"14.4 ms^{-1}" = "72/5 ms{-1}"$.}% výsledek ve všech možných formách včetně veličiny % TODO cisla
{uhlirova}% autor vzoráku~--~typicky prijmeni (jinak o tom vis), vsechna mala
{\opt{lngcs}{% řešení CZ
Z druhého grafu vyčteme na y-ové ose celkovou dráhu ($"2\,160 m"$) a na x-ové celkový čas
($"150 s"$). Průměrná rychlost tedy vychází $"2\,160/150 = 14.4 ms^{-1}" = "72/5 ms{-1}"$.
}%
\opt{lngsk}{% riešenie SK
}}%
{\opt{lngcs}{}\opt{lngsk}{}}% pôvod úlohy
% TODO obrazok
