% Neodmazavejte ty % jinak vam Donald ruce ukousne!!!
%
% Bylo by super, když se úloha zásadně přejmenuje, tak přejmenovat
% i její soubor a opravit to v problem-sort.tex
%
\uloha{\opt{lngcs}{kolečka}\opt{lngsk}{}}% název s malým písmenem (velké u vl. jmen)
{\opt{lngcs}{% zadání CZ
Patrik si zapojil několik ozubených koleček jako na obrázku (čísla v kolečkách
označují počet zubů) a levé kolečko roztočil na rychlost $"10 ot/min"$.
Jakou rychlostí se bude otáčet pravé kolečko?
}%
\opt{lngsk}{% zadanie SK
}}%
{Pravé kolečko se otáčí rychlostí $"25 ot/min"$.}% výsledek ve všech možných formách včetně veličiny
{uhlirova}% autor vzoráku~--~typicky prijmeni (jinak o tom vis), vsechna mala
{\opt{lngcs}{% řešení CZ
Stačí si uvědomit, že za stejný čas se na každém ozubeném kolečku musí otočit
stejný počet zubů. Počty zubů koleček uprostřed tedy nemají vliv na rychlost
posledního kolečka a zajímá nás jen poměr zubů prvního a posledního kolečka.
Tento poměr je $25/10 = "2.5"$, poslední kolečko se tedy otáčí $"2.5"$-krát
rychleji, tedy rychlostí $"25 ot/min"$.
}%
\opt{lngsk}{% riešenie SK
}}%
{\opt{lngcs}{}\opt{lngsk}{}}% pôvod úlohy
