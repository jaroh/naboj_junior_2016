% Neodmazavejte ty % jinak vam Donald ruce ukousne!!!
%
% Bylo by super, když se úloha zásadně přejmenuje, tak přejmenovat
% i její soubor a opravit to v problem-sort.tex
%
\uloha{\opt{lngcs}{plavání}\opt{lngsk}{}}% název s malým písmenem (velké u vl. jmen)
{\opt{lngcs}{% zadání CZ
Peťa s Petrem si šli jednou zaplavat. Peťa uplave jednu délku bazénu za $"54 s"$,
pomalejšímu Petrovi trvá uplavání stejné vzdálenosti $"63 s"$. Za jak dlouho se setkají
na tom samém kraji bazénu, pokud vyrazí společně?
}%
\opt{lngsk}{% zadanie SK
}}%
{Znovu se setkají za $"756 s" = "12 min 36 s"$.}% výsledek ve všech možných formách včetně veličiny
{uhlirova}% autor vzoráku~--~typicky prijmeni (jinak o tom vis), vsechna mala
{\opt{lngcs}{% řešení CZ
Aby se plavci potkali, každý z nich musí v tom samém časde uplavat několik
přesných délek bazénu. Hledáme tedy nejmenší společný násobek čísel $54$ a $63$,
kterým je $378$. Peťa však za $"378 s"$ uplave $"378 s"/"54 s" = 7$ délek
a Petr $6$. To znamená, že v danou chvíli se budou nacházet na opačných koncích
bazénu. Na stejném konci se setkají za dvojnásobek tohoto času, tedy za
$"756 s" = "12 min 36 s"$.
}%
\opt{lngsk}{% riešenie SK
}}%
{\opt{lngcs}{}\opt{lngsk}{}}% pôvod úlohy
