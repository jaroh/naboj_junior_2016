% Neodmazavejte ty % jinak vam Donald ruce ukousne!!!
%
% Bylo by super, když se úloha zásadně přejmenuje, tak přejmenovat
% i její soubor a opravit to v problem-sort.tex
%
\uloha{\opt{lngcs}{kreslíme do obdélníku}\opt{lngsk}{}}% název s malým písmenem (velké u vl. jmen)
{\opt{lngcs}{% zadání CZ
Terka si na papír nakreslila obdélník s délkami stran $"8 cm"$ a $"6 cm"$. Podél
jeho delší strany vkreslila dovnitř obdélníku trojúhelník o stranách $"3 cm"$ a
$"7 cm"$. Vrchol tohoto trojúhelníku spojila se zbylými dvěma vrcholy obdélníku
(viz obrázek). Jaký je obsah vyznačené šedé oblasti?
}%b
\opt{lngsk}{% zadanie SK
}}%
{Obsah vyznačené šedé oblasti je $"24 cm^2"$.}% výsledek ve všech možných formách včetně veličiny
{uhlirova}% autor vzoráku~--~typicky prijmeni (jinak o tom vis), vsechna mala
{\opt{lngcs}{% řešení CZ
Celý obdélník si můžeme rozdělit na čtyři menší tak, že každý je z poloviny tvořen
šedou oblastí (viz obrázek). Odtud vidíme, že obsah šedé oblasti je roven polovině
obsahu celého obdélníku, tedy $1/2 \cdot (6 \cdot 8) = "24 cm^2"$.
}%
\opt{lngsk}{% riešenie SK
}}%
{\opt{lngcs}{}\opt{lngsk}{}}% pôvod úlohy
