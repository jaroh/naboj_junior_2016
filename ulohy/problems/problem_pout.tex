% Neodmazavejte ty % jinak vam Donald ruce ukousne!!!
%
% Bylo by super, když se úloha zásadně přejmenuje, tak přejmenovat
% i její soubor a opravit to v problem-sort.tex
%
\uloha{\opt{lngcs}{pouťové atrakce}\opt{lngsk}{}}% název s malým písmenem (velké u vl. jmen)
{\opt{lngcs}{% zadání CZ
Děti na letním táboře šly na výlet do přilehlé vesnice, kde se konala pouť.
$70$ z nich si zašlo na řetízkový kolotoč, $75$ si šlo zajezdit na autodrom,
$85$ si zvolilo horskou dráhu a $80$ dětí dovádělo na skákacím hradě.
Kolik nejméně dětí vyzkoušelo všechny zmíněné pouťové atrakce, jestliže
jich na táboře bylo celkem sto?
}%
\opt{lngsk}{% zadanie SK
}}%
{Všechny atrakce si vyzkoušelo deset dětí.}% výsledek ve všech možných formách včetně veličiny
{uhlirova}% autor vzoráku~--~typicky prijmeni (jinak o tom vis), vsechna mala
{\opt{lngcs}{% řešení CZ
U každé atrakce si dopočteme, kolik dětí ji nevyzkoušelo: $30$ dětí nevyzkoušelo
řetízkový kolotoč, $25$ autodrom, $15$ horskou dráhu a $20$ skákací hrad.
Hledáme nejmenší počet dětí, které vyzkoušely všechny atrakce, proto uvažujeme,
že nějakou z atrakcí vynechalo pokaždé jiné dítě. Všechny atrakce pak muselo
navštívit nejméně $100 - \(30 + 25 + 20 + 15\) = 10$ dětí.
}%
\opt{lngsk}{% riešenie SK
}}%
{\opt{lngcs}{}\opt{lngsk}{}}% pôvod úlohy
