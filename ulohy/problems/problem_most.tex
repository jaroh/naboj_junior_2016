% Neodmazavejte ty % jinak vam Donald ruce ukousne!!!
%
% Bylo by super, když se úloha zásadně přejmenuje, tak přejmenovat
% i její soubor a opravit to v problem-sort.tex
%
\uloha{\opt{lngcs}{Kolínský most}\opt{lngsk}{}}% název s malým písmenem (velké u vl. jmen)
{\opt{lngcs}{%
Nový zvedací most v Kolíně váží rovných $"150 t"$ a měří $"10 m"$ (viz obrázek).
Zvednutí mostu z vodorovné do svislé polohy trvá $"8 min"\,"20 s"$. Lukáš během
této doby hravě spočítal průměrný výkon zdvihacího zařízení. Spočítejte ho také.
\opt{lngsk}{% zadanie SK
}}%
{Výkon zdvihacího zařízení je $"15\,000 W" = "15 kW"$.}% výsledek ve všech možných formách včetně veličiny
{svancara}% autor vzoráku~--~typicky prijmeni (jinak o tom vis), vsechna mala
{\opt{lngcs}{% řešení CZ
První zjevná komplikace spočívá v tom, že síla, kterou je potřeba zvedat most
se v čase mění. Místo sil tedy použijeme energie, neboť víme, že při zvedání
se změní potenciální energie mostu o $mgh$, kde $m$ je hmotnost mostu, $g$
tíhové zrychlení a $h$ je změna výšky \emph{těžiště} mostu. Tato energie se
musí rovnat práci zdvihacího zařízení, kterou vyjádříme jako součin jeho výkonu
$P$ a času zvedání $t = "8 min"\,"20 s" = "500 s"$. Platí tedy
\eq{
  Pt = mgh\,, \ztoho P = \frac{mgh}{t}\,.
}
Jelikož těžiště se nachází v polovině délky mostu, při zdvihání se změní jeho výška
o $h = "5 m"$. Po dosazení zbylých údajů již snadno dostaneme, že výkon
zdvihacího zařízení je $P = "15\,000 W" = "15 kW"$.
% TODO v zadani neni napsano, ze most je homogenni
}%
\opt{lngsk}{% riešenie SK
}}%
{\opt{lngcs}{}\opt{lngsk}{}}% pôvod úlohy
% TODO obrázek
