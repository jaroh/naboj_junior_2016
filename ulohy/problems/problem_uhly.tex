% Neodmazavejte ty % jinak vam Donald ruce ukousne!!!
%
% Bylo by super, když se úloha zásadně přejmenuje, tak přejmenovat
% i její soubor a opravit to v problem-sort.tex
%
\uloha{\opt{úhly}{}\opt{lngsk}{}}% název s malým písmenem (velké u vl. jmen)
{\opt{lngcs}{% zadání CZ
Vypočítejte velikost úhlu \alpha v nákresu na obrázku. Střed kružnice
\emph{leží} na naznačené přímce.}%
\opt{lngsk}{% zadanie SK
}}%
{Hledaný úhel je $\alpha = "20\dg"$.}% výsledek ve všech možných formách včetně veličiny
{uhlirova}% autor vzoráku~--~typicky prijmeni (jinak o tom vis), vsechna mala
{\opt{lngcs}{% řešení CZ
V obrázku si vyznačíme úhly $\beta$, $\gamma$ a $\delta$ (viz obrázek). Z vedlejších
úhlů získáme $\gamma = 180 - 118 = "62\dg"$. Pomocí souhlasných a středových úhlů 
máme $\alpha = "48\dg"$. Ze součtu vnitřních úhlů trojúhelníku $"180\dg"$ zjistíme
$\beta = 180 - (62 + 48) = "70\dg"$. Dále si všimneme Thaletovy kružnice, tedy
trojúhelník ABC je pravoúhlý. Odtud nám vychází $\alpha = 180 - (90 + \beta) = "20\dg"$.
}%
\opt{lngsk}{% riešenie SK
}}%
{\opt{lngcs}{}\opt{lngsk}{}}% pôvod úlohy
% TODO obrázek
