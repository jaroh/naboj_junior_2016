% Neodmazavejte ty % jinak vam Donald ruce ukousne!!!
%
% Bylo by super, když se úloha zásadně přejmenuje, tak přejmenovat
% i její soubor a opravit to v problem-sort.tex
%
\uloha{\opt{lngcs}{zlomky, zlomky, zlomky\cdot}\opt{lngsk}{}}% název s malým písmenem (velké u vl. jmen)
{\opt{lngcs}{% zadání CZ
Andreu by zajímalo, kolik je $1/2$ z $2/3$ z $3/4$ z $4/5$ z $5/6$ z $6/7$
z $7/8$ z $8/9$ z $9/10$ z čísla $"1\,000"$?
}%
\opt{lngsk}{% zadanie SK
}}%
{Hledané číslo je $100$.}% výsledek ve všech možných formách včetně veličiny
{uhlirova}% autor vzoráku~--~typicky prijmeni (jinak o tom vis), vsechna mala
{\opt{lngcs}{% řešení CZ
Hledáme číslo $x$, pro které platí
\eq{
  x = \frac{1}{2} \cdot \frac{2}{3} \cdot \frac{3}{4} \cdot \frac{4}{5}
\cdot \frac{5}{6} \cdot \frac{6}{7} \cdot \frac{7}{8} \cdot \frac{8}{9} \cdot
\frac{9}{10} \cdot "1\,000"\,.
}
Všimněme si, že lze pokrátit jmenovatele a čitatele dvou po sobě jdoucích
zlomků. Zůstane nám tedy pouze čitatel prvního ($1$) a jmenovatel posledního
($10$) zlomku. Hledané číslo je tedy $x = "1\,000"/10 = 100$.
}%
\opt{lngsk}{% riešenie SK
}}%
{\opt{lngcs}{}\opt{lngsk}{}}% pôvod úlohy
