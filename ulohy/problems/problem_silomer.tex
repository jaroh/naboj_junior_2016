% Neodmazavejte ty % jinak vam Donald ruce ukousne!!!
%
% Bylo by super, když se úloha zásadně přejmenuje, tak přejmenovat
% i její soubor a opravit to v problem-sort.tex
%
\uloha{\opt{lngcs}{siloměry}\opt{lngsk}{}}% název s malým písmenem (velké u vl. jmen)
{\opt{lngcs}{% zadání CZ
Katka našla v školní laboratoři dva siloměry. Oba měli v nenataženém stavu
délku $"15 cm"$, ale první měl dílky odpovídacící síle $"1 N"$ dlouhé $"1 cm"$,
zatím co druhý $"3 cm"$. Katka pak mezi siloměry připevnila závaží o hmotnosti
$"700 g"$ a tloušťce $"5 cm"$ a napjala je mezi dvě vodorovné desky vzdálené
$"50 cm"$ (viz obrázek). Jakou sílu ukazoval první siloměr, pověšený seshora?
}%
\opt{lngsk}{% zadanie SK
}}%
{Siloměr pověšený seshora ukazoval $"9 N"$.}% výsledek ve všech možných formách včetně veličiny
{}% autor vzoráku~--~typicky prijmeni (jinak o tom vis), vsechna mala
{\opt{lngcs}{% řešení CZ
}%
\opt{lngsk}{% riešenie SK
}}%
{\opt{lngcs}{}\opt{lngsk}{}}% pôvod úlohy
