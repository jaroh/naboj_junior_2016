% Neodmazavejte ty % jinak vam Donald ruce ukousne!!!
%
% Bylo by super, když se úloha zásadně přejmenuje, tak přejmenovat
% i její soubor a opravit to v problem-sort.tex
%
\uloha{\opt{lngcs}{kde je nula}\opt{lngsk}{}}% název s malým písmenem (velké u vl. jmen)
{\opt{lngcs}{%
Ondra viděl v metru napsaný mnohočlen $x^5 - 7x^4 + 3x^3 + 4x^2 - 16x - 17$.
Kamarád Vojta mu prozradil, že tento mnohočlen má právě jeden kořen
($x$ pro které je hodnota mnohočlenu rovna nule) na intervalu
$\left\langle{0;10}\right\rangle$. Zjistěte hodnotu tohoto kořene s přesností $\pm "0.5"$.
}%
\opt{lngsk}{% zadanie SK
}}%
{Kořen mnohočlenu se v rámci tolerance nachází v intervalu $\left\langle{"6.02";"7.02"}\right\rangle$.}% výsledek ve všech možných formách včetně veličiny
{}% autor vzoráku~--~typicky prijmeni (jinak o tom vis), vsechna mala
{\opt{lngcs}{% řešení CZ
}%
\opt{lngsk}{% riešenie SK
}}%
{\opt{lngcs}{}\opt{lngsk}{}}% pôvod úlohy
