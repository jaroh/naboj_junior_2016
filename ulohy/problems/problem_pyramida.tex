% Neodmazavejte ty % jinak vam Donald ruce ukousne!!!
%
% Bylo by super, když se úloha zásadně přejmenuje, tak přejmenovat
% i její soubor a opravit to v problem-sort.tex
%
\uloha{\opt{lngcs}{pyramida}\opt{lngsk}{}}% název s malým písmenem (velké u vl. jmen)
{\opt{lngcs}{% zadání CZ
Trojúhelník $\bod{ABC}$ je rozdělený třemi přímkami na čtyři části, přičemž
přímky jsou rovnoběžné a navzájem od sebe, strany $\bod{BC}$ a vrcholu $\bod{A}$
rovnoměrně rozmístěné. Tom ví, že obsah druhé části zdola (šedě na obrázku) je
$"35 cm^2"$, a tak si hned spočítal obsah celého trojúhelníku. Kolik mu to vyšlo?
}%
\opt{lngsk}{% zadanie SK
}}%
{Obsah celého trojúhelníku je $"112 cm^2"$.}% výsledek ve všech možných formách včetně veličiny
{}% autor vzoráku~--~typicky prijmeni (jinak o tom vis), vsechna mala
{\opt{lngcs}{% řešení CZ
Jelikož úsečky, které dělí trojúhelník $\bod{ABC}$ jsou rovnoběžné s $\bod{AB}$,
vzniklé menší trojúhelníky se společným bodem $\bod{C}$ mají stejné vnitřní
úhly jako $\bod{ABC}$ a jsou si tedy podobné.

Z rovnoměrného rozmístění dělících úseček vyplývá, že výšky zmíněných
trojúhelníků procházejících bodem $\bod{C}$ jsou v poměru ${1:2:3:4}$.
Z podobnosti pak plyne, že ve stejném poměru musí být i délky dělících úseček
a strana $\bod{AB}$. Jelikož dělící úsečky odpovídají základnám trojúhelníků
a obsah trojúhelníků závisí na součinu délek základen a výšek, plochy menších
trojúhelníků a $\bod{ABC}$ musí být v poměru ${1^2 : 2^2 : 3^2 : 4^2} = {1:4:9:16}$.

Označíme-li plochu nejmenšího trojúhelníku jako jeden dílek, správný poměr ploch
zbylých trojúhelníků dostaneme tehdy, když plochy jednotlivých pásů budou
mít postupně $1$, $3$, $5$ a $7$ dílků. Ze známé plochy šedého dílku pak určíme,
že $1\,\text{dílek} = "35 cm^2"/5 = "7 cm^2"$. Plocha trojúhelníku $\bod{ABC}$
odpovídá $16$ dílkům, tedy ploše $16 \cdot "7 cm^2" = "112 cm^2"$.
}%
\opt{lngsk}{% riešenie SK
}}%
{\opt{lngcs}{}\opt{lngsk}{}}% pôvod úlohy
