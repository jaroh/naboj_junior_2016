% Neodmazavejte ty % jinak vam Donald ruce ukousne!!!
%
% Bylo by super, když se úloha zásadně přejmenuje, tak přejmenovat
% i její soubor a opravit to v problem-sort.tex
%
\uloha{\opt{lngcs}{blyštivá}\opt{lngsk}{}}% název s malým písmenem (velké u vl. jmen)
{\opt{lngcs}{% 
Julie přednedávnem našla krystalek fluoridu vápenatého (\ce{CaF2}) o objemu
$"1 cm^3"$. Ví, že krystal má kubickou strukturu -- molekuly
\ce{CaF2} jsou uspořádány do krychlové sítě (viz obrázek) tak, že atomy vápníku
(větší kuličky) se nacházejí v rozích krychle a ve středech stěn,
atomy fluoru (menší kuličky) tvoří krychli o poloviční délce hrany umístěné
uprostřed té velké.

Délka strany fluorové krychle je $"5 \AA"$. V tabulkách si našla,
že jeden atom vápníku má hmotnost $"65e{-27} kg"$ a flouru $"30e{-27} kg"$.
Jakou hmotnost má Juliin krystal?
}%
\opt{lngsk}{% zadanie SK
}}%
{Juliin krystal má hmotnost $"2e{-3} kg" = "2 g"$.}% výsledek ve všech možných formách včetně veličiny
{}% autor vzoráku~--~typicky prijmeni (jinak o tom vis), vsechna mala
{\opt{lngcs}{% řešení CZ
}%
\opt{lngsk}{% riešenie SK
}}%
{\opt{lngcs}{}\opt{lngsk}{}}% pôvod úlohy
% TODO urcit rozumnou presnost
% TODO dopsat do tabulek \AA
