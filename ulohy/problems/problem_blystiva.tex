% Neodmazavejte ty % jinak vam Donald ruce ukousne!!!
%
% Bylo by super, když se úloha zásadně přejmenuje, tak přejmenovat
% i její soubor a opravit to v problem-sort.tex
%
\uloha{\opt{lngcs}{blyštivá}\opt{lngsk}{}}% název s malým písmenem (velké u vl. jmen)
{\opt{lngcs}{% 
Petra má ráda blyštivé věci. Před nedávnem našla krystal fluoridu vápenatého (\ce{CaF2})
o objemu $"1 cm^{-3}"$. Ví, že fluorid vápenatý má pravidelnou kubickou strukturu -- molekuly
\ce{CaF2} jsou uspořádány do krychlové sítě, jak je vidět na obrázku (větší kuličky představují vápník, menší fluor). Délka strany této krychle (vzdálenost dvou atomů vápníku) je $"5 \AA"$. V tabulkách si našla,
že jeden atom vápníku má hmotnost $"65e{-27} kg"$ a flouru $"30e{-27} kg"$.
Jakou hmotnost má přibližně Petřin krystal?
}%
\opt{lngsk}{% zadanie SK
}}%
{}% výsledek ve všech možných formách včetně veličiny
{}% autor vzoráku~--~typicky prijmeni (jinak o tom vis), vsechna mala
{\opt{lngcs}{% řešení CZ
}%
\opt{lngsk}{% riešenie SK
}}%
{\opt{lngcs}{}\opt{lngsk}{}}% pôvod úlohy
% TODO urcit rozumnou presnost
% TODO dopsat do tabulek \AA
