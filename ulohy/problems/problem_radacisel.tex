% Neodmazavejte ty % jinak vam Donald ruce ukousne!!!
%
% Bylo by super, když se úloha zásadně přejmenuje, tak přejmenovat
% i její soubor a opravit to v problem-sort.tex
%
\uloha{\opt{lngcs}{řada čísel}\opt{lngsk}{}}% název s malým písmenem (velké u vl. jmen)
{\opt{lngcs}{% zadání CZ
Lukáš napsal za sebou osm (ne nutně různých) čísel, přičemž součet každých
třech po sobě jdoucích je $42$. Třetí číslo zleva je $15$ a osmé zleva $19$.
Jaké číslo je úplně vlevo?
}%
\opt{lngsk}{% zadanie SK
}}%
{Úplně vlevo je číslo $8$.}% výsledek ve všech možných formách včetně veličiny
{uhlirova}% autor vzoráku~--~typicky prijmeni (jinak o tom vis), vsechna mala
{\opt{lngcs}{% řešení CZ
Hledané číslo si označíme $x$. Z podmínky součtu tří po sobě jdoucích čísel 
získáme, že na druhém místě musí být číslo $42 - \(15 + x\) = 27 - x$.
Podobně pro další čísla nám vyjde $x$ (čtvrté), $27 - x$ (páté), $15$ (šesté).
Pro sedmé zleva pak máme opět $x$,kde konečně ze součtu poslední trojice
$19 + 15 + x = 42$ dopočítáme $x = 8$.
}%
\opt{lngsk}{% riešenie SK
}}%
{\opt{lngcs}{}\opt{lngsk}{}}% pôvod úlohy
