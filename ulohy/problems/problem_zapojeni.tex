% Neodmazavejte ty % jinak vam Donald ruce ukousne!!!
%
% Bylo by super, když se úloha zásadně přejmenuje, tak přejmenovat
% i její soubor a opravit to v problem-sort.tex
%
\uloha{\opt{lngcs}{nekonečné zapojení}\opt{lngsk}{}}% název s malým písmenem (velké u vl. jmen)
{\opt{lngcs}{% zadání CZ
Nejlepší, co se dá s nekonečnou zásobou rezistorů o odporech $"1 \Ohm"$ a
$"2 \Ohm"$ udělat, je zapojit je podle obrázku do nekonečného zapojení.
Vypočítejte celkový odpor tohoto zapojení! Výsledek udejte s přesností na jedno
desetinné místo.
}%
\opt{lngsk}{% zadanie SK
}}%
{Celkový odpor zapojení je $"\sqrt{3} + 1 \Omega" \approx 2.7 \Omega"$ }% výsledek ve všech možných formách včetně veličiny
{hofierka}% autor vzoráku~--~typicky prijmeni (jinak o tom vis), vsechna mala
{\opt{lngcs}{% řešení CZ
Uvažovaná síť rezistorů je nekonečná, proto přidáme-li k této síti ještě jeden „kousek“, bude jich tam stále nekonečně mnoho a celkový odpor se nezmění. To nám umožní celé zapojení překreslit do velmi jednoduché náhradní sítě, jejíž odpor musí být roven celkovému odporu původní nekonečné sítě (viz obrázek).

Stačí si vzpomenout na pravidlá pro počítaní sériového a paralelního zapojení rezistorů a při označení menšího z rezistorů ($"1 \Ohm"$) $"R"$ a většího $"2R"$ píšeme rovnost:
\eq{
  R_c = 2R + \left(\frac{1}{\frac{1}{R}+\frac{1}{R_c}}\right) = 2R + \left(\frac{R_c R}{R_c + R}\right),
}
kde $R_c$ značí hledaný celkový odpor zapojení. Nyní upravme rovnici násobením faktorem $(R_c + R)$ a dostáváme:
\eq{
  R_c^2 + R_c R = 2R^2 + 2R_c R + R_c R,
}
což vede na kvadratickou rovnici:
\eq{
  0 = R_c^2 - 2R_c R - 2R^2 = (R_c - R)^2 - 3R^2,
}
kterou však umíme vyřešit pouhým odmocněním a úpravou: $"R_c - R = \sqrt{3} R"$. Tedy celkový odpor zapojení je $"\sqrt{3} + 1 \Omega" \approx 2.7 \Omega"$.
}%
\opt{lngsk}{% riešenie SK
}}%
{\opt{lngcs}{}\opt{lngsk}{}}% pôvod úlohy
% TODO obrázek prekresleneho zapojeni do vzoráku