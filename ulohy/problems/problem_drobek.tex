% Neodmazavejte ty % jinak vam Donald ruce ukousne!!!
%
% Bylo by super, když se úloha zásadně přejmenuje, tak přejmenovat
% i její soubor a opravit to v problem-sort.tex
%
\uloha{\opt{lngcs}{drobek}\opt{lngsk}{}}% název s malým písmenem (velké u vl. jmen)
{\opt{lngcs}{% zadání CZ
Na náměstí širokém $"10 m"$ stojí budova vysoká $"8 m"$. 
Na okraji střechy této budovy stojí holub a mlsně pozoruje drobek rohlíku, který leží
někde mezi ním a opačným koncem náměstí.Na druhé straně
náměstí, přesně na proti holubovi stojí na zemi hrdlička
s úmyslem sníst tentýž drobek. Ve stejnou chvíli
vyrazí holub i hrdlička k drobku a to tou nejkratší možnou cestou. Pohybují
se stejnou rychlostí a oba dva se u drobku střetnou v tentýž čas. Vypočtěte,
jakou vzdálenost holub uletěl během cesty za drobkem.
}%
\opt{lngsk}{% zadanie SK
}}%
{Holub uletěl vzdálenost $"8.2 m"$.}% výsledek ve všech možných formách včetně veličiny
{uhlirova}% autor vzoráku~--~typicky prijmeni (jinak o tom vis), vsechna mala
{\opt{lngcs}{% řešení CZ
Hrdlička i holub letěli k drobku stejně dlouho touž rychlostí, tedy jejich vzdálenosti
od drobku se rovnají a označíme je $x$. Drobek, věž a holub tvoří pravoúhlý trojúhelník,
pro který máme Pythagorovu větu $8^2 + (10-x)^2 = x^2$, kterou snadno upravíme na
$20x = 164$, tedy $x = "8.2 m"$.
}%
\opt{lngsk}{% riešenie SK
}}%
{\opt{lngcs}{}\opt{lngsk}{}}% pôvod úlohy
