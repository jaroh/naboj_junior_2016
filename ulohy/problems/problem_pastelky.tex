% Neodmazavejte ty % jinak vam Donald ruce ukousne!!!
%
% Bylo by super, když se úloha zásadně přejmenuje, tak přejmenovat
% i její soubor a opravit to v problem-sort.tex
%
\uloha{\opt{lngcs}{pastelky}\opt{lngsk}{}}% název s malým písmenem (velké u vl. jmen)
{\opt{lngcs}{% zadání CZ
Kolika způsoby může Borek uspořádat šest pastelek ve svém penále (červenou,
oranžovou, žlutou, zelenou, modrou, fialovou), aby žlutá a zelená nebyly
vedle sebe?
}%
\opt{lngsk}{% zadanie SK
}}%
{Pastelky může uspořádat $480$ různými způsoby.}% výsledek ve všech možných formách včetně veličiny
{}% autor vzoráku~--~typicky prijmeni (jinak o tom vis), vsechna mala
{\opt{lngcs}{% řešení CZ
Kdyby Borek neměl omezení na sousedství žluté a zelené pastelky, počet možných
uložení lze vypočíst jako součin možností uložení pastelek: první pastelku
lze uložit do penálu na šest míst, druhou na zbylých pět, třetí na čtyři atd.
čímž dostaneme $6 \cdot 5 \cdot 4 \cdot 3 \cdot 2 \cdot 1 = 720$ možností.

Pokusme se teď zjistit počet nevyhovujících případů, tzn. kolik uspořádání
má zelenou a žlutou pastelku vedle sebe. Dvojici pastelek lze umístit do
penálu na celkem $5$ pozic, tzn. $10$ různými způsoby (zelenou a žlutou
můžeme na každé z pozic prohodit). Zbylé čtyři pastelky pak lze umístit
$4 \cdot 3 \cdot 2 \cdot 1 = 24$ způsoby na zbylé volné pozice v penálu.
Nevyhovujících možností je tedy $10 \cdot 24 = 240$ a vyhovujících
$720 - 240 = 480$.

Borek tedy může umístit do penálu pastelky $480$ různými způsoby.
}%
\opt{lngsk}{% riešenie SK
}}%
{\opt{lngcs}{}\opt{lngsk}{}}% pôvod úlohy
