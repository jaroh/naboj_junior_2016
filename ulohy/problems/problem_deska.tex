% Neodmazavejte ty % jinak vam Donald ruce ukousne!!!
%
% Bylo by super, když se úloha zásadně přejmenuje, tak přejmenovat
% i její soubor a opravit to v problem-sort.tex
%
\uloha{\opt{lngcs}{zrádné síly}\opt{lngsk}{}}% název s malým písmenem (velké u vl. jmen)
{\opt{lngcs}{% zadání CZ
Homogenní deska dlouhá $"5 m"$ a vážicí $"20 kg"$ je podepřena dvěma podpěrami,
a to na levém konci a ve vzdálenosti $"3 m"$ od levého konce. Na pravý konec
desky umístíme závaží o hmotnosti $"5 kg"$. Jakou sílou (a jakým směrem)
působí deska na levou podpěru?
}%
\opt{lngsk}{% zadanie SK
}}%
{Deska na levou podpěru nepůsobí žádnou silou}% výsledek ve všech možných formách včetně veličiny
{hofierka}% autor vzoráku~--~typicky prijmeni (jinak o tom vis), vsechna mala
{\opt{lngcs}{% řešení CZ
Pakliže deska není v pohybu, je výslednice sil na ní působícich rovna nule. Tedy tíhová síla (působící v těžišti desky) a síla od závaží se kompenzují sílami opačného smeru od obou podpěr. Analýza působícich sil (tedy strejda Newton a jeho zákony) nám ale neodhalí, v jaké míře jsou ve spomínané kompenzaci zastoupeny jednotlivé podpěry.

Pakliže deska nekoná rotační pohyb, je výslednice momentů sil na ní působícich vzhledem k libovolnému bodu rovna nule.
Zvolme momentový bod (tj. bod, vůči kterému počítáme momenty) vrchol pravé podpěry (abychom eliminovali moment síly mezi deskou a pravou podpěrou). Moment tíhové síly od závaží má velikost $"M_z= 5 kg \cdot 10 ms^{-2} \cdot 2 m"$ a moment tíhové síly od desky má opačný směr a velikost $"M_d= 20 kg \cdot 10 ms^{-2} \cdot 0,5 m"$ (protože působiště tíhové síly je v těžišti desky). Na první pohled vidíme, že $"M_d=M_z"$, jinými slovy, momenty se akorát vyruší a pro levou podpěru nezbývá, než nepůsobit na desku žádným momentem a protože rameno je nenulové, nulová musí být síla. A to je řešení této úlohy. } %
\opt{lngsk}{% riešenie SK
}}%
{\opt{lngcs}{}\opt{lngsk}{}}% pôvod úlohy
