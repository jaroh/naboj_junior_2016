% Neodmazavejte ty % jinak vam Donald ruce ukousne!!!
%
% Bylo by super, když se úloha zásadně přejmenuje, tak přejmenovat
% i její soubor a opravit to v problem-sort.tex
%
\uloha{\opt{lngcs}{mokrá kostka}\opt{lngsk}{}}% název s malým písmenem (velké u vl. jmen)
{\opt{lngcs}{%
Petr pozoruje kostku plovoucí na vodní hladině. Svým odborným
pohledem určí, že nad hladinu vyčnívá přesně $"20 \%"$. Posléze
zjistí, že k úplnému ponoření kostky na ni musí působit kolmo svisle
silou o velikosti $"3 N"$. Jaký je objem kostky?
}%
\opt{lngsk}{% zadanie SK
}}%
{Objem kostky je $"1.5 \ell" = "1.5 dm^3" = "0.001\,5 m^3"$.}% výsledek ve všech možných formách včetně veličiny
{svancara}% autor vzoráku~--~typicky prijmeni (jinak o tom vis), vsechna mala
{\opt{lngcs}{% řešení CZ
V případě volně plovoucí kostky je v rovnováze vztlaková síla $\rho V' g$
($\rho$ je hustota vody, $V' = "0.8"V$ je objem ponořené části a $g$ tíhové
zrychlení) a tíhová síla $m g = \rho\_k V g$ ($\rho\_k$ je hustota kostky).
Srovnáním těchto sil dostaneme hustotu kostky:
\eq{
  "0.8" \rho V g = \rho\_h V g\,, \ztoho \rho\_k = "0.8"\rho = "800 kg/m^3"\,.
}
Působení sily $F = "3 N"$ ve směru stejném jako je směr tíhové síly vyvolá
ponoření kostky, tedy i změnu ponořeného objemu na $V$.
Rovnost sil se tedy změní na
\eq{
  mg + F = \rho\_k V g + F = \rho V g\,.
}
Úpravou rovnice dostaneme
\eq{
  F = \(\rho - \rho\_k\) V g\,, \ztoho V = \frac{F}{\(\rho - \rho\_k\) g} = "0.0015 m^3" = "1.5 \ell"\,.
}
Objem kostky je $"1.5 \ell"$.
}%
\opt{lngsk}{% riešenie SK
}}%
{\opt{lngcs}{}\opt{lngsk}{}}% pôvod úlohy
