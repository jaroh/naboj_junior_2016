% Neodmazavejte ty % jinak vam Donald ruce ukousne!!!
%
% Bylo by super, když se úloha zásadně přejmenuje, tak přejmenovat
% i její soubor a opravit to v problem-sort.tex
%
\uloha{\opt{lngcs}{uhlopříčky}\opt{lngsk}{}}% název s malým písmenem (velké u vl. jmen)
{\opt{lngcs}{% zadání CZ
Evu by zajímalo, který pravidelný mnohoúhelník má $54$ uhlopříček. Který?
}%
\opt{lngsk}{% zadanie SK
}}%
{Hledaný mnohoúhelník je dvanáctiúhelník.}% výsledek ve všech možných formách včetně veličiny
{}% autor vzoráku~--~typicky prijmeni (jinak o tom vis), vsechna mala
{\opt{lngcs}{% řešení CZ
Postupujme od nejjednodušších útvarů. Po nakreslení zjistíme, že trojúhelník
nemá uhlopříčku, čtverec je má dvě, pětiúhelník pět, šestiúhelník devět\ldots
Takto můžeme pokračovat i dále, no je výhodnější si všimnout, že rozdíl
v počtu uhlopříček vzroste mezi dvěma nasledujícími monohoúhelníky vždy o jednu.

Bez dalšího kreslení tak můžeme určit, že sedmiúhelník má $14$ uhlopříček,
osmiúhelník $20$, devětiúhelník $27$, desetiúhelník $35$, jedenáctiúhelník
$44$ a dvanáctiúhelník $54$ uhlopříček.
}%
\opt{lngsk}{% riešenie SK
}}%
{\opt{lngcs}{}\opt{lngsk}{}}% pôvod úlohy
