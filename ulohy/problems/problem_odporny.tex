% Neodmazavejte ty % jinak vam Donald ruce ukousne!!!
%
% Bylo by super, když se úloha zásadně přejmenuje, tak přejmenovat
% i její soubor a opravit to v problem-sort.tex
%
\uloha{\opt{lngcs}{odporný NÁBOJ}\opt{lngsk}{}}% název s malým písmenem (velké u vl. jmen)
{\opt{lngcs}{% 
Radka se nudila při hodině fyziky, a tak si z rezistorů, každý s odporem
$"10 \Ohm"$, sestavila nápis \uv{NABOJ}, jak je znázorněno na obrázku. Poté mezi
zvýrazněnými body změřila odpor a sestavila písmenka vedle sebe od nejmenší
naměřené hodnoty po největší. Jaké \uv{slovo} získala?
}%
\opt{lngsk}{% zadanie SK
}}%
{Radka získala slovo OBAJN. Odpory jsou (v pořadí): $70/8" \Ohm" = 8 + \frac{6}{8} " \Ohm"$,
$40/3" \Ohm" = 13 + \frac{1}{3} " \Ohm"$, $80/3" \Ohm" = 26 + \frac{2}{3} " \Ohm"$,
$"40 \Ohm"$ a $"60 \Ohm"$.}% výsledek ve všech možných formách včetně veličiny
{}% autor vzoráku~--~typicky prijmeni (jinak o tom vis), vsechna mala
{\opt{lngcs}{% řešení CZ
V řešení využijeme vztahy pro odpor sériového a paralelního zapojení. Počítejme
postupně. V písmenu N je sériově zapojených $8$ rezistorů, takže jeho odpor
je $R\_N = "10 \Ohm" + "10 \Ohm" + \cdots + "10 \Ohm" = "80 \Ohm"$. Ve středě
písmena A je paralelní zapojení dvou a jednoho rezistoru. Odpor této trojice
$R_3$ vypočítáme ze vztahu pro paralelní zapojení
\eq{
  \frac{1}{R_3} = \frac{1}{"10 \Ohm"} + \frac{1}{"20 \Ohm"} = \frac{3}{"20 \Ohm"}\,, \ztoho R_3 = \frac{20}{3}" \Ohm"\,.
}
Celkový odpor písmena A je pouhé sériové zapojení dvou rezistorů a části
s odporem $R_3$, tedy
\eq{
  R\_A = "10 \Ohm" + \frac{20}{3}" \Ohm" + "10 \Ohm" = \frac{80}{3}" \Ohm"\,.
}
V písmenu B jsou síriově zapojené dvě části s odporem $R_3$, tedy
\eq{
  R\_B = \frac{20}{3}" \Ohm" + \frac{20}{3}" \Ohm" = \frac{40}{3}" \Ohm"\,.
}
V písmenu O jsou dvě paralelní větve, jedna s oporem $"10 \Ohm"$ a druhá, delší,
s odporem $"70 \Ohm"$. Platí tedy
\eq{
  \frac{1}{R\_O} = \frac{1}{"10 \Ohm"} + \frac{1}{"70 \Ohm"} = \frac{8}{"70 \Ohm"}\,, \ztoho R\_O = \frac{70}{8}" \Ohm"\,.
}
Konečně v písmenu N jsou sériově zapojené $4$ rezistory s celkovým odporem
$R\_J = "40 \Ohm"$. Seřazením odporů podle velikosti dostáváme
$R\_O < R\_B < R\_A < R\_J < R\_N$. Hledané \uv{slovo} je tedy OBAJN.
}%
\opt{lngsk}{% riešenie SK
}}%
{\opt{lngcs}{}\opt{lngsk}{}}% pôvod úlohy
