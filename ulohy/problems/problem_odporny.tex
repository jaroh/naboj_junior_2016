% Neodmazavejte ty % jinak vam Donald ruce ukousne!!!
%
% Bylo by super, když se úloha zásadně přejmenuje, tak přejmenovat
% i její soubor a opravit to v problem-sort.tex
%
\uloha{\opt{lngcs}{odporný NÁBOJ}\opt{lngsk}{}}% název s malým písmenem (velké u vl. jmen)
{\opt{lngcs}{% 
Radka se nudila na praktikách z elektřiny a tak si z rezistorů, každý s odporem
$"10 \Ohm"$, sestavila nápis \uv{NABOJ}, jak je znázorněno na obrázku. Poté mezi
zvýrazněnými body změřila odpor a sestavila písmenka vedle sebe od nejmenší
naměřené hodnoty po největší. Jaké \uv{slovo} získala?
}%
\opt{lngsk}{% zadanie SK
}}%
{Radka získala slovo OBAJN. Odpory jsou (v pořadí): $7/8" \Ohm"$, $4/3" \Ohm"$, $8/3" \Ohm"$, $"4 \Ohm"$ a $"6 \Ohm"$.}% výsledek ve všech možných formách včetně veličiny
{}% autor vzoráku~--~typicky prijmeni (jinak o tom vis), vsechna mala
{\opt{lngcs}{% řešení CZ
}%
\opt{lngsk}{% riešenie SK
}}%
{\opt{lngcs}{}\opt{lngsk}{}}% pôvod úlohy
