% Neodmazavejte ty % jinak vam Donald ruce ukousne!!!
%
% Bylo by super, když se úloha zásadně přejmenuje, tak přejmenovat
% i její soubor a opravit to v problem-sort.tex
%
\uloha{\opt{lngcs}{osmiúhelník}\opt{lngsk}{}}% název s malým písmenem (velké u vl. jmen)
{\opt{lngcs}{% zadání CZ
Terku by zajímalo, kolik pravoúhlých trojúhelníků může vytvořit tak, že vrcholy
těchto trojúhelníků se budou nacházet ve vrcholech pravidelného osmiúhelníku. Kolik?
}%
\opt{lngsk}{% zadanie SK
}}%
{Terka může vytvořit $24$ pravoúhlých trojúhelníků.}% výsledek ve všech možných formách včetně veličiny
{uhlirova}% autor vzoráku~--~typicky prijmeni (jinak o tom vis), vsechna mala
{\opt{lngcs}{% řešení CZ
V osmiúhelníku můžeme najít 4 obdélníky, jejichž dvě protilehlé strany se shodují
s protilehlými stranami osmiúhelníku, a 2 čtverce, jejichž vrcholy jsou totožné 
s vrcholy osmiúhelníku (viz obrázek). Každý z těchto čtyřúhelníků můžeme rozříznout podél
dvou uhlopříček, tedy z jednoho čtyřúhelníku vzniknou 4 pravoúhlé trojúhelníky. Celkem
máme tedy $(4+2) \cdot 4 = 24$ pravoúhlých trojúhelníků.
}%
\opt{lngsk}{% riešenie SK
}}%
{\opt{lngcs}{}\opt{lngsk}{}}% pôvod úlohy
