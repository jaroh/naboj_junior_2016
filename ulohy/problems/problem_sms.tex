% Neodmazavejte ty % jinak vam Donald ruce ukousne!!!
%
% Bylo by super, když se úloha zásadně přejmenuje, tak přejmenovat
% i její soubor a opravit to v problem-sort.tex
%
\uloha{\opt{lngcs}{prodlení signálu}\opt{lngsk}{}}% název s malým písmenem (velké u vl. jmen)
{\opt{lngcs}{% zadání CZ
}%
\opt{lngsk}{% zadanie SK
}}%
{Signál, který vysílají sondy zkoumající povrch Marsu přichází na Zem s jistým
spožděním, protože se signál šíří pouze konečnou rychlostí $"300\,000 km/s"$
(rychlost světla ve vakuu). V průběhu roku se však doba spoždění mění
v rozmezí $"250 s"$ až $"1\,250 s"$. V jaké vzdálenosti od Slunce obíhá Mars?
Předpokládejte, že obě planety obíhají kolem Slunce po kruhových dráhach.
}% výsledek ve všech možných formách včetně veličiny
{}% autor vzoráku~--~typicky prijmeni (jinak o tom vis), vsechna mala
{\opt{lngcs}{% řešení CZ
}%
\opt{lngsk}{% riešenie SK
}}%
{\opt{lngcs}{}\opt{lngsk}{}}% pôvod úlohy
