% Neodmazavejte ty % jinak vam Donald ruce ukousne!!!
%
% Bylo by super, když se úloha zásadně přejmenuje, tak přejmenovat
% i její soubor a opravit to v problem-sort.tex
%
\uloha{\opt{lngcs}{hodina chemie}\opt{lngsk}{}}% název s malým písmenem (velké u vl. jmen)
{\opt{lngcs}{% zadání CZ
Hanka měla při hodině chemie před sebou dvě kádinky. 
V první z nich bylo $"15 ml"$ třicetiprocentního lihu a ve druhé 
$"35 ml"$ padesátiprocentního roztoku stejné látky.
Kolikaprocentní roztok lihu získá, když celý obsah obou kádinek smíchá?
(Kontrakci objemu nevažujte.)
}%
\opt{lngsk}{% zadanie SK
}}%
{Hanka získá smícháním $"44\%"$ roztok lihu.}% výsledek ve všech možných formách včetně veličiny
{uhlirova}% autor vzoráku~--~typicky prijmeni (jinak o tom vis), vsechna mala
{\opt{lngcs}{% řešení CZ
Čistého lihu je $0.3 \cdot 15 = "4.5 ml"$ v první kádince a $0.5 \cdot 35 = "17.5 ml"$
v druhé. Po smíšení obou kádinek získáme $4.5 + 17.5 = "22 ml"$ lihu v $15 + 35 = "50 ml"$
roztoku, tedy $"22/50 \cdot 100\%" = "44\%"$ roztok lihu.
Alternativně si vzpomeneme na směšovací rovnici $w_1 V_1 + w_2 V_2 = w_3 V_3$,
dosazením do které získáme stejný výsledek.
}%
\opt{lngsk}{% riešenie SK
}}%
{\opt{lngcs}{}\opt{lngsk}{}}% pôvod úlohy
