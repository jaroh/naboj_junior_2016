% Neodmazavejte ty % jinak vam Donald ruce ukousne!!!
%
% Bylo by super, když se úloha zásadně přejmenuje, tak přejmenovat
% i její soubor a opravit to v problem-sort.tex
%
\uloha{\opt{lngcs}{hodina chemie}\opt{lngsk}{}}% název s malým písmenem (velké u vl. jmen)
{\opt{lngcs}{% zadání CZ
Hanka měla při hodině chemie před sebou dvě kádinky. 
V první z nich bylo $"15 m\ell"$ třicetiprocentního lihu a ve druhé 
$"35 m\ell"$ padesátiprocentního roztoku stejné látky.
Kolikaprocentní roztok lihu získá, když celý obsah obou kádinek smíchá?
(Kontrakci objemu nevažujte.)
}%
\opt{lngsk}{% zadanie SK
}}%
{Hanka získá smícháním $"44\%"$ roztok lihu.}% výsledek ve všech možných formách včetně veličiny
{uhlirova}% autor vzoráku~--~typicky prijmeni (jinak o tom vis), vsechna mala
{\opt{lngcs}{% řešení CZ
Čistého lihu je $"0.3" \cdot "15 m\ell" = "4.5 m\ell"$ v první kádince
a $"0.5" \cdot "35 m\ell" = "17.5 m\ell"$ v druhé. Po smíšení obou kádinek
získáme $"4.5 m\ell" + "17.5 m\ell" = "22 m\ell"$ lihu
v $"15 m\ell" + "35 m\ell = "50 m\ell"$ roztoku. Koncentrace roztoku je tedy 
$"22 m\ell"/"50 m\ell" \cdot "100 \%" = "44 \%"$. Hanka smíchaním získá
$"44\%"$ roztok lihu.
}%
\opt{lngsk}{% riešenie SK
}}%
{\opt{lngcs}{}\opt{lngsk}{}}% pôvod úlohy
