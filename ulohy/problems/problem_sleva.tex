% Neodmazavejte ty % jinak vam Donald ruce ukousne!!!
%
% Bylo by super, když se úloha zásadně přejmenuje, tak přejmenovat
% i její soubor a opravit to v problem-sort.tex
%
\uloha{\opt{lngcs}{Petr šetří}\opt{lngsk}{}}% název s malým písmenem (velké u vl. jmen)
{\opt{lngcs}{% zadání CZ
Petr dostal dvě slevové poukázky do svého oblíbeného obchodu. První poukázka byla
na slevu $"40 \%"$ na libovolný kus oblečení a druhá na zlevnění libovolného kusu 
oblečení na $"99 \Kc"$. V obchodě se Petrovi zalíbila košile za $"150 \Kc"$. 
Se kterou poukázkou ušetří víc a o kolik více ušetří?
}%
\opt{lngsk}{% zadanie SK
}}%
{Se $"40\%"$ slevou ušetří o $"9 \Kc"$ více.}% výsledek ve všech možných formách včetně veličiny
{uhlirova}% autor vzoráku~--~typicky prijmeni (jinak o tom vis), vsechna mala
{\opt{lngcs}{% řešení CZ
S poukázkou $"40\%"$ slevy tričko stojí o $"0.40 \cdot 150 = 60 \Kc"$ méně,
s druhou poukázkou o $"150 - 99 = 51 \Kc"$. Výhodnější je tedy použít poukázku
se slevou $"40 \%"$, čímž Petr ušetří o $"9 \Kc"$ více.
}%
\opt{lngsk}{% riešenie SK
}}%
{\opt{lngcs}{}\opt{lngsk}{}}% pôvod úlohy
